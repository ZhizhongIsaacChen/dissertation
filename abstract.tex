\abstract{  
\todo{Write abstract.}
}
  %Thanks to technology advancements, data are nowadays being collected for
  %almost all human activities and many natural phenomena, to the point that the
  %challenge in many disciplines is the \emph{analysis} of data rather than the
  %collection of it. 

%  The huge volume, high variety of origins and formats, and
%  ever-increasing velocity requested when reporting results, make the
%  task of analyzing the data a complex computational problem known as \emph{Big
%  Data Analytics}. Traditional techniques developed in statistics are often
%  insufficient, lacking scalability (\emph{volume}), speed (\emph{velocity}), or
%  applicability (\emph{variety}). The thesis of this work is that it is possible
%  to address these issues by using modern techniques from \emph{statistical learning
%  theory} to develop and analyze of efficient randomized algorithms for
%  extracting collections of interesting patterns from large transactional
%  datasets and huge graphs. Problems of interest include frequent itemsets and
%  association rules mining, frequent subgraphs extraction, and database queries
%  selectivity estimation. The two main lines of research involve exploiting the
%  trade-off between \emph{speed} and \emph{accuracy} by developing algorithms
%  that only analyze small random samples of the original data, and assessing the
%  \emph{statistical significance} of the extracted patterns by creating
%  statistical tests that can be applied in a multiple hypothesis testing
%  setting. The analysis of the algorithms uses tools from statistical learning
%  theory like \emph{VC-dimension}, \emph{shatter coefficients}, and
%  \emph{Rademacher averages}. A byproduct of this thesis is the evidence that
%  these tools, often considered only of theoretical interest, can be of great
%  use in practice. For each proposed algorithm, an extensive empirical
%  evaluation assess its performances on real and artificial benchmark datasets,
%  showing that their output is often of quality higher than what the theoretical
%  analysis could guarantee.

