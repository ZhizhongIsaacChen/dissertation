\abstract{  
Analyzing huge datasets becomes prohibitively slow when the dataset does not
fit in main memory. Approximations of the results of guaranteed high quality
are sufficient for most applications and can be obtained very fast by analyzing
a small random part of the data that fits in memory. We study the use of the
Vapnik-Chervonenkis dimension theory to analyze the trade-off between the sample
size and the quality of the approximation for fundamental problems in knowledge
discovery (frequent itemsets), graph analysis (betweenness centrality), and
database management (query selectivity). 
\\
We show that the sample size to compute an high-quality approximation of
the collection of frequent itemsets depends only on the VC-dimension of the
problem, which is (tightly) bounded from above by an easy-to-compute characteristic
quantity of the dataset. This bound leads to a fast algorithm for mining
frequent itemsets that we also adapt to the MapReduce framework for
parallel/distributed computation. We exploit similar ideas to avoid the
inclusion of false positives in mining results.
\\
The betweenness centrality index of a vertex in a network measures the
relative importance of that vertex by counting the fraction of shortest paths
going through that vertex. We show that it is possible to compute an
high-quality approximation of the betweenness of all the vertices by sampling
shortest paths at random. The sample size depends on the VC-dimension of the
problem, which is upper bounded by the logarithm of the maximum number of
vertices in a shortest path. The tight bound collapses to a constant when there
is a unique shortest path between any two vertices.
\\
The selectivity of a database query is the ratio between the size of its output
and the product of the sizes of its input tables. Database Management Systems
estimate the selectivity of queries for scheduling and optimization purposes. We
show that it is possible to bound the VC-dimension of queries in terms of their
SQL expressions, and to use this bound to compute a sample of the database that
allow much a more accurate estimation of the selectivity than possible using
histograms. 
}

