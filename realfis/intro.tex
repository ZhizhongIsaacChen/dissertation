%The extraction of association rules is one of the fundamental primitives in
%data mining and knowledge discovery from large databases~\citep{AgrawalIS93}.
%In its most general definition, the problem can be reduced to identifying
%frequent sets of items, or \emph{frequent itemsets}, appearing in at
%least a fraction  $\theta$ of all transactions in a dataset, where $\theta$ is provided in
%input by the user. Frequent itemsets and association rules are not only of
%interest for classic data mining applications (e.g., market basket analysis), but
%are also useful for further data analysis and mining task, including clustering,
%classification, and indexing~\citep{han2006data,HanCXY07}.

In this chapter we present the task of mining Frequent Itemsets (FIs) from a
different point of view, and show that we can use VC-dimension to avoid the
inclusion of \emph{false positives} in the results of mining. In most
applications, the set of FIs is not interesting \emph{per
se}. %
%one is not interested in mining a dataset to extract the
%frequent itemsets \emph{per se}. 
Instead, the mining results %process is 
are used to infer properties of the \emph{underlying process} that generated the
dataset. Consider for example the following scenario: a team of researchers would like %wants 
to identify frequent associations (i.e., itemsets) between preferences among
Facebook users. To this end, they set up an online survey %on Facebook 
which is filled out by a \emph{small fraction} of Facebook users (some users may even
take the survey multiple times). Using this information, the researchers want to
infer the associations (itemsets) that are frequent for the \emph{entire} Facebook
population. In fact, the %one can 
 whole Facebook population and the online survey define the underlying \emph{process} that
generated the dataset \emph{observed} by the researchers. We are
interested in answering the following question: %and ask 
how can we use the latter (the observed dataset) to identify itemsets that are
frequent in the former (the whole population)?
%from which the dataset observed by the researcher has been
%generated, and wants to use the latter to identify frequent itemsets in the
%former. 
This is a very natural question, as is the underlying assumption that the
observed dataset is \emph{representative} of the generating process. For
example, in market basket analysis, 
%Another example is given by market basket analysis: in this case 
%one uses the %current, 
the observed purchases of customers are used to infer the future
purchase habits %pattern 
of all customers while assuming that %in the future 
the purchase behavior that generated the dataset is representative of the one
that will be followed in the future.
%that generated the current dataset %will be followed).

A natural and general model to describe these concepts %settings 
is to assume that the transactions in the dataset $\Ds$ are \emph{independent
identically distributed} (i.i.d.) samples from an \emph{unknown} probability
distribution $\prob$ defined on all possible transactions built on a set of
items. %with items appearing in $\Ds$. %built on the sets of items appearing $\Ds$. 
Since $\prob$ is fixed, each itemset $A$ has a fixed \emph{probability} $\tfreq(A)$
to appear in a transaction sampled from $\prob$. We call $\tfreq(A)$ the
\emph{true frequency} of $A$ (w.r.t.~$\prob$). The true frequency corresponds
to the fraction of transactions that would contain the itemset $A$ among an
hypothetical infinite
set %number 
of transactions. %, that contain the itemset $A$. % and each itemset has a fixed probability to appear in a random transaction from $\prob$. 
The real goal of the mining process is then to identify itemsets that have
true frequency $\tfreq$ at least $\theta$. We call such itemsets the \emph{True Frequent
Itemsets} (TFIs). %to appear in a random transaction drawn from $\prob$, given that for an itemset such
%probability corresponds to the fraction of transactions, among an infinite
%number of transactions, that contain the itemset. 
In the market basket analysis example%above
, $\Ds$ contains %is 
the observed purchases of customers, the \emph{unknown} distribution $\prob$
describes the purchase behavior of the customers as a whole, and we want to
analyze $\Ds$ to find the itemsets that have probability (i.e., true frequency) %$\tfreq$ 
at least $\theta$ to be bought by a customer. Note that we made no assumption on
$\prob$, except from the fact that the transactions in the dataset are
i.i.d.~samples from $\prob$. This is in contrast to other settings that assume
that the generative distribution $\prob$ is such that items appear in
transactions generated by $\prob$ totally or partially independently from each
other.

Since $\Ds$ represents only a \emph{finite} sample from $\prob$, the set $F$ of frequent itemsets
of $\Ds$ w.r.t.~$\theta$ %at threshold $\theta$ 
only provides an \emph{approximation} of the True Frequent Itemsets: %, and 
due to the stochastic nature of the generative process, the set 
$F$ %of frequent itemsets of $\Ds
may contain a number of \emph{false positives}, %\emph{spurious (or false) discoveries}
 i.e., itemsets that appear among the frequent itemsets of
 $\Ds$ but whose \emph{true} frequency is smaller than %are not generated with probability at least 
$\theta$. %, may be reported. 
At the same time, some itemsets with true frequency
greater than $\theta$ may have a frequency in $\Ds$ that is \emph{smaller} than
$\theta$ (\emph{false negatives}), and therefore not be in $F$. This implies
that %Given that $\prob$ is not known, on one hand 
one can not aim at identifying \emph{all and only} the itemsets having true frequency %probability 
at least $\theta$.
% in $\prob$.
Even worse, from the data analyst's point of view, % On the other hand, 
there is \emph{no guarantee or bound on the number of false positives} reported
in $F$. %the collection of frequent itemsets of $\Ds$ w.r.t.~$\theta$.
%using the frequent itemsets of $\Ds$ as a proxy for such itemsets does not
%provide any \emph{guarantee on the number of false discoveries}. %that are reported. 
Consider the following scenario as an example. Let $A$ and $B$ be two (disjoint)
sets of pairs of items. The set $A$ contains 1,000 disjoint pairs, while $B$
contains 10,000 disjoint pairs. Let $\prob$ be such that, for any pair $(a,a')\in
A$, we have $\tfreq((a,a'))=0.1$, and for any pair $(b,b')\in B$, we have
$\tfreq((b,b'))=0.09$. Let $\Ds$ be a dataset of 10,000 transactions sampled from
$\prob$. %For example, assume that %in $\prob$ 
%there are two sets $A$ and $B$ of pairs of items:
%the set $A$ of 1000 (disjoint) pairs of items with  probability 0.1 in $\prob$, and
%the set $B$ of 10000 (disjoint) pairs of items with probability 
%0.09 in $\prob$. Assume that all pairs are independent in $\prob$, and we have 
%a dataset $\Ds$ of 10000 transactions from $\prob$ .
%If 
We are interested in finding pairs of items that have true frequency at least %probability 
$\theta=0.095$. %in $\prob$, and
If we extract the pairs of items with frequency at least $\theta$ in $\Ds$,
it is easy to see that in expectation %around 
50 of the 1,000 pairs from $A$ will have frequency in $\Ds$ %appear in $\Ds$ with frequency
\emph{below} $0.095$, and in expectation %around 
400 pairs from $B$ will have frequency in $\Ds$ %appear in $\Ds$ with frequency 
\emph{above} $0.095$.
Therefore, the set of pairs that have frequency at least $\theta$ in $\Ds$ does
\emph{not} contain % are missing 
some of the pairs that have true frequency %probability 
at least $\theta$ %in $\prob$ 
(false negatives), but %and also 
includes a huge number of %some 
pairs that have true frequency smaller than $\theta$ (false positives). %below $\theta$ in $\prob$ (false positives or discoveries).

In general, one would like to avoid false positives and 
at the same time find as many TFIs as possible. These are somewhat contrasting
goals, and care must be taken to achieve a good balance between them.
A na\"ive but \emph{overly conservative} method to avoid false positives
involves the use of \emph{Chernoff and union bounds}~\citep{MitzenmacherU05}.
Given an itemset $A$ in $\Ds$, the quantity $|\Ds|f_\Ds(A)$ is a random variable with
Binomial distribution $\mathcal{B}(|\Ds|,\tfreq(A))$. It is possible to use
standard methods like the Chernoff and the union bounds to bound the deviation
of the frequencies in the dataset of \emph{all} itemsets from their
expectations. These tools can be used to compute a value $\hat\theta$ such that
the probability that a non-true frequent itemset $B$ has frequency
greater or equal to $\hat\theta$ is at most $1-\delta$, for some
$\delta\in(0,1)$. This method has the following serious drawback: in order to
achieve such guarantee, it is \emph{necessary} to bound the deviation of the
frequencies of \emph{all itemsets possibly appearing in the dataset}~\citep{KirschMAPUV12}. This means
that, if the transactions are built on a set of $n$ items, the union bound must
be taken over all $2^n-1$ potential itemsets, even if some or most of them may
appear with very low frequency or not at all in samples from $\prob$. As a
consequence, the chosen value of $\hat\theta$ is extremely \emph{conservative}, despite
being sufficient to avoid the inclusion of false positives in mining results.  %Indeed the 
The collection of itemsets with frequency at least $\hat\theta$ in $\Ds$,
although consisting (probabilistically) only of TFIs, only contains a
\emph{very small} portion of them, due to the overly conservative choice of
$\hat\theta$. (The results of our experimental evaluation
in Sect.~\ref{sec:experiments} clearly show the limitations of this method.) More
refined algorithms are therefore needed to achieve the correct balance between
the contrasting goals of avoiding false positives and finding as many TFIs as
possible.

%The problem of identifying the itemsets that have true frequency %appear with probability 
%at least $\theta$ with guarantees on the quality of the returned set has
%received scant attention in the literature. 

\paragraph*{Our contributions.}
The contributions we make are the following:
\begin{itemize}
  \item We formally define the problem of mining the \emph{True Frequent
    Itemsets} w.r.t.~a minimum threshold $\theta$, and we develop and analyze an
    algorithm to \emph{identify a value $\hat{\theta}$ such that, with
    probability at least $1-\delta$, all itemsets
with frequency at least $\hat{\theta}$ in the dataset have true frequency
at least $\theta$}. Our method is completely \emph{distribution-free}, i.e., it
does not make \emph{any} assumption about the unknown generative distribution
$\prob$. %We only need to assume that the transactions are independent samples
%from the distribution $\prob$, without any constraint on the properties of
%$\prob$. 
By contrast, existing methods to assess the significance of frequent patterns after their
extraction %This is in contrast with the assumptions that are made by methods that assess
%the significance of the frequent patterns after they have been identified: these
%methods 
require a well specified, limited generative model to characterize the
significance of a pattern. When %Moreover, when %On the other hand, if 
additional information about the distribution $\prob$ is available, it can be
incorporated in our method to obtain even higher accuracy.
\item %mathematical tools 
We analyze our algorithm using results from \emph{statistical learning theory} and \emph{optimization}. %to develop and analyze our algorithm. 
We define a range space associated to a collection of itemsets and give an upper
bound to its (empirical) VC-dimension and a procedure to compute this bound,
showing an interesting connection with the Set-Union
Knapsack Problem (SUKP)~\citep{GoldschmidtNY94}. 
%This generalizes results from~\citet{RiondatoU12}.
To the best of our knowledge, we are the first to apply these
techniques to the field of TFIs, and in general the first application of the
sample complexity bound based on \emph{empirical} VC-dimension
to the field of data mining. 
\item We implemented our algorithm and assessed its performances on simulated
  datasets with properties -- number of items, itemsets frequency distribution,
  etc.-- similar to real datasets. We computed the fraction of TFIs contained in the set of frequent itemsets in
  $\Ds$ w.r.t.~$\hat\theta$, and the number of false positives, if any. The
  results show %We noticed 
  that the algorithm is even \emph{more accurate} than the theory guarantees, since \emph{no
  false positive} %discovery 
is reported in any of the many experiments we performed,
  and moreover allows the \emph{extraction of almost all TFIs}. %: only a very small fraction of the TFIs is not included in the output collection. 
  We also
compared the set of itemsets computed by our method to those obtained with the
``Chernoff and union bounds'' method presented in the introduction, and found
that our algorithm \emph{vastly outperforms} it.
 %of the binomial distribution, and to the results of
%``vanilla'' mining of the dataset $\Ds$ at frequency $\theta$.
\end{itemize}

%In this work %paper 
%we define and address the problem of identifying the \emph{True
%Frequent Itemsets} (TFIs ) with respect to a minimum frequency threshold
%$\theta$ %, that is the itemsets that appear with probability at least $\theta$, % in $\prob$,
%while providing \emph{rigorous probabilistic guarantees} on the number of false
%discoveries, and \emph{without making any assumption} on the generative model of
%the transactions (i.e., on the distribution $\prob$). This makes the method we
%introduce completely \emph{distribution free}. 
%%We develope our methods within the statistical hypothesis testing framewe are
the firstwork: 
%%each itemset has an associated null hypothesis claiming that the itemset has
%%probability less than $\theta$. This hypothesis is tested using information
%%obtained from the dataset, and accepted or rejected accordingly. If the
%%hypothesis is rejected, the itemsets is included in the output collection.
%%in which
%%each itemset is associated with a null hypothesis corresponding to the itemset
%%being generated with probability less than $\theta$.
%
%In particular, we develop and analyze an algorithm to \emph{identify a threshold
%$\hat{\theta}$ such that all itemsets with frequency at least $\hat{\theta}$
%in the dataset have true frequency %probability 
%at least $\theta$}. %in $\prob$; 
%This guarantee is \emph{probabilistic}, in the sense that it holds with probability at
%least $1 - \delta$, for some $\delta\in(0,1)$ specified by the user. (In the
%language of hypothesis testing, we provide a method that returns a set of TFIs
%with Family-Wise Error Rate -- FWER -- bounded by %the user-specified limit
%$\delta$.)
%
%We use results %mathematical tools 
%from \emph{statistical learning theory} and \emph{optimization} to develop and
%analyze our algorithm. We define a range space associated to a collection of
%itemsets and give an upper bound to its (empirical) VC-dimension and a procedure
%to compute this bound, showing an interesting connection with the Set-Union
%Knapsack Problem (SUKP)~\citep{GoldschmidtNY94}. 
%%This generalizes results from~\citet{RiondatoU12}.
%To the best of our knowledge, ours is the first work to apply these
%techniques to the field of TFIs, and in general the first application of the
%sample complexity bound based on \emph{empirical} VC-Dimension and of the SUKP
%to the field of data mining. 
%
%We stress that we do not impose \emph{any restriction} on the generative model of the
%transactions that are observed in the transactional dataset: our algorithm is
%completely \emph{distribution-free}. In fact, we only assume that the
%transactions are independent samples from the distribution $\prob$, without any
%constraint on the properties of $\prob$. This is in contrast with the
%assumptions that are made by methods that assess the significance of the
%frequent patterns after they have been identified. In fact, these methods
%require a well specified, limited generative model to characterize the
%significance of a pattern. On the other hand, if additional information about
%the distribution $\prob$ is available, it can be incorporated in our method to
%obtain even higher accuracy.
%% while this is usually not possible with traditional
%% methods.
%
%We implemented our algorithm and evaluate its performances (on simulated datasets 
%with properties -number of items, itemsets frequency distribution, etc.- similar to 
%real datasets)
%by analyzing
%the set of itemsets with frequency in $\Ds$ at least $\hat{\theta}$: we computed
%the fraction of TFIs contained in it and the number of false discoveries, if
%any. We noticed that the algorithm is even more accurate than the theory
%guarantees and allows the extraction of almost all TFIs: only a very small
%fraction of the TFIs is not included in the output collection. Moreover, no
%false discovery is reported in any of the many experiments we performed. We also
%compared the set of itemsets computed by our method to those obtained with a
%method based on the bounds of the binomial distribution, and to the results of
%``vanilla'' mining of the dataset $\Ds$ at frequency $\theta$.
%%Lastly, we tested whether the frequency in the dataset is a good
%%estimator for the real frequency of a TFI, answering this questions positively.
%%\XXX: only if
%%we include these results. \MR 

\paragraph*{Outline.} %The article is organized as follows. 
In Sect.~\ref{sec:prevwork} we review relevant previous contributions.
Sections ~\ref{sec:prelims} and~\ref{sec:range} contain preliminaries to
formally define the problem and key concepts that we will use throughout the
work. Our proposed algorithm is described and analyzed in Sect.~\ref{sec:main}.
We present the methodology and results of our experimental evaluation %of the test 
in Sect.~\ref{sec:experiments}. Conclusions and future directions can be found %are presented 
in Sect.~\ref{sec:concl}. 

\iffalse
This is an example of a statistical test, in which the probability, or $p$-value,
that a measure, or \emph{statistic} (in the example, the frequency of $\{a,b\}$)
is at least as extreme as the value observed in real data is computed under a
\emph{null hypothesis} that captures the properties of spurious discoveries (in
the example, the independence of items). When the $p$-value is small enough, the
itemset is flagged as significant, otherwise the itemset is discarded as a
spurious discovery. A number of different procedures~\citep{SilversteinBM98,MegiddoS98,DuMouchelP01,GionisMMT07,Hamalainen10,KirschMAPUV12} have been proposed
in recent years to control the number of spurious discoveries that are reported
\emph{after} the frequent itemsets are identified. These procedures take into
account the fact that a transactional dataset contains a number of patterns, and
the assessment of their significance therefore is a \emph{multiple
hypothesis testing} problem.


For example, consider
the transactions given by items that are bought together on Amazon; after
observing a certain number of transactions, one is interested in inferring
association between items that are valid for the distribution over \emph{all} possible
purchases, not only for the current, observed set $\Ds$ of purchases that
represents only a partial observation obtained from the distribution that
includes also purchases that have not been recorded in the dataset, or that will
materialize only in the future.

The itemsets
that are not frequent in $p$ but are frequent in $\Ds$ are false discoveries, and are not going to be filtered by
the statistical tests described above, since such tests \emph{assume} that the
itemsets that are frequent in $\Ds$, whose significance they assess, represent the frequent itemsets
in $p$ as well. In fact, these tests define itemsets as spurious by considering
properties of the itemsets other than its frequency. In the example above, the
co-occurrence of $\{a,b\}$ is likely not due to random chance; however, the
probability that $\{a,b\}$ appears with frequency $3\%$  in $\Ds$ while its
frequency in $p$ is $1\%$ is $0.08$, therefore if $\theta=2\%$ then $\{a,b\}$ likely is a false
discovery. As noted by~\citet{LiuZW11}, the phase of assessment of the
significance of the frequent itemsets cannot replace the role played by the minimum support
threshold $\theta$, that is to reflect the level of domain significance, and is to be used
in concert with statistical significance to filter uninteresting patterns that
arise from different sources. Therefore being able to rigorously identify frequent patterns in $p$ 
is crucial in order to obtain high quality patterns.

In this paper we address the problem of
identifying itemsets that appear with probability at least $\theta$ in $p$,
that we call \emph{True Frequent Itemsets} (RFI), while providing rigorous
probabilistic guarantees on the number of false discoveries, without making any
assumption on the particular generative model of the transactions. This makes
our method completely \emph{distribution free}. In particular, we focus on
returning a set of RFI with bounded Family-Wise Error Rate (FWER), that is the
probability that one or more false discovery is reported among the RFI. A
recently proposed alternative to  bound the FWER is to bound the False Discovery
Rate (FDR), that in our case correspond to the proportion of false discoveries
among the RFI. The use of the FDR allows to produce in output a larger number of
patterns, since a small proportion of false discoveries are tolerated in the output;
however, in data mining the number of patterns produced is usually high,
therefore having a smaller number of high quality discoveries is preferable to
reporting a larger number of patterns containing some false discoveries.
\fi

\iffalse
{\bf XXX:} we should explain here what a statistical test is, what the FWER,
what the difference with the FDR, and so on. If we do it well here, we probably
do not need to do it again the preliminaries.

{\bf XXX:} We should really stress that we are not mining ``statistically
significant'' itemsets, as this would imply some kind of underlying model (e.g.,
``items appear independently in transactions'') that instead we do not have. I
believe we should actually somewhat comment on such models, which are clearly
too simplistic to be really useful/meaningful: in real/natural data generation
processes itemsets \emph{clearly} do not appear independently, so comparing the
dataset to such a model is only of limited value. 
\fi


