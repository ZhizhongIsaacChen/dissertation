\section{Experimental evaluation}\label{sec:experiments}
We conducted an extensive evaluation to assess the performances of the algorithm
we propose. In particular, we used it %our algorithm 
to compute values $\hat\theta$
for a number of frequencies $\theta$ on different datasets, and compared the
collection of FIs w.r.t.~$\hat\theta$ with the collection of TFIs, measuring
the number of false positives %discoveries 
and the fraction of TFIs that were found.

%practical applicability and
%the statistical power of the methods we propose. In the following sections we
%describe the methodology and present the results. 
%Due to space limitations,
%only a portion of our experimental results are reported here. The full report,
%including evaluations of the computational efficiency of the methods is
%available in the full version of the paper~\citep{RiondatoV13}.

\begin{table*}[tbp]
  \centering
  \begin{tabular}{llrrr}
\toprule
Dataset & Freq.~$\theta$ & TFIs & Times FPs & Times FNs\\
\midrule
\texttt{accidents} & 0.2 & 889883 & 100\% & 100\% \\
%& 0.8 & 149 & 0 & 0 & 0.0 & 0 & 0 & 0 & 0.0 & 0 \\
% & 0.7 & 529 & 1 & 0 & 0.1 & 2 & 4 & 0 & 0.7 & 7 \\
% & 0.6 & 2074 & 7 & 0 & 1.75 & 14 & 4 & 0 & 0.5 & 7 \\
% & 0.5 & 8057 & 20 & 0 & 8.05 & 19 & 8 & 0 & 3.65 & 17 \\
% & 0.45 & 16123 & 28 & 1 & 10.1 & 20 & 14 & 1 & 6.35 & 20 \\
% & 0.4 & 32528 & 70 & 4 & 26.55 & 20 & 46 & 2 & 16.55 & 20 \\
% & 0.35 & 68222 & 160 & 14 & 59.85 & 20 & 94 & 9 & 39.0 & 20 \\
% & 0.3 & 149545 & 313 & 16 & 132.4 & 20 & 180 & 10 & 79.15 & 20 \\
% & 0.25 & 346525 & 761 & 73 & 319.75 & 20 & 378 & 33 & 189.3 & 20 \\
% & 0.2 & 889883 & 1988 & 184 & 837.5 & 20 & 961 & 81 & 470.55 & 20 \\
% & 0.15 & 2676381 & 6370 & 0 & 2194.6 & 16 & 1786945 & 364 & 358398.9 & 20 \\
\midrule
\texttt{BMS-POS} & 0.005 & 4240 & 100\% & 100\% \\
%\multirow{6}{*}{\texttt{BMS-POS}} & 0.05 & 59 & 0 & 0 & 0.0 & 0 & 1 & 0 & 0.3 & 6 \\
% & 0.03 & 134 & 1 & 0 & 0.05 & 1 & 0 & 0 & 0.0 & 0 \\
% & 0.02 & 308 & 2 & 0 & 0.85 & 14 & 2 & 0 & 0.2 & 3 \\
% & 0.01 & 1099 & 4 & 1 & 1.45 & 20 & 6 & 3 & 3.85 & 20 \\
% & 0.0075 & 1896 & 10 & 2 & 4.85 & 20 & 13 & 4 & 7.7 & 20 \\
% & 0.005 & 4240 & 16 & 3 & 8.85 & 20 & 31 & 11 & 20.95 & 20 \\
\midrule
\texttt{chess} & 0.6 & 254944 & 100\% & 100\% \\
%\multirow{4}{*}{\texttt{chess}} %& 0.95 & 77 & 0 & 0 & 0.0 & 0 & 0 & 0 & 0.0 & 0 \\
% %& 0.9 & 622 & 1 & 0 & 0.05 & 1 & 0 & 0 & 0.0 & 0 \\
% & 0.8 & 8227 & 0 & 0 & 0.0 & 0 & 23 & 0 & 7.45 & 17 \\
% & 0.75 & 20993 & 0 & 0 & 0.0 & 0 & 131 & 12 & 68.25 & 40 \\
% & 0.65 & 111239 & 192 & 9 & 72.4 & 20 & 181 & 0 & 16.55 & 15 \\
% & 0.6 & 254944 & 193 & 1 & 45.5 & 20 & 875 & 1 & 155.1 & 20 \\
% & 0.5 & 1272932 & 80 & 0 & 13.1 & 9 & 4767 & 1481 & 2815.0 & 18 \\
\midrule
\texttt{connect} & 0.85 & 142127 & 100\% & 100\% \\
%\multirow{3}{*}{\texttt{connect}} %& 0.975 & 327 & 0 & 0 & 0.0 & 0 & 2 & 0 & 0.8 & 8 \\
% %& 0.95 & 2201 & 12 & 0 & 2.5 & 11 & 8 & 0 & 2.8 & 10 \\
% & 0.9 & 27127 & 64 & 0 & 12.8 & 10 & 96 & 0 & 28.6 & 16 \\
% & 0.875 & 65959 & 240 & 0 & 63.6 & 18 & 208 & 0 & 62.4 & 19 \\
%& 0.85 & 142127 & 384 & 16 & 145.6 & 20 & 328 & 32 & 182.0 & 20 \\
\midrule
\texttt{kosarak} & 0.015 & 189 & 45\% & 55\% \\
%\multirow{6}{*}{\texttt{kosarak}} & 0.04 & 42 & 0 & 0 & 0.0 & 0 & 0 & 0 & 0.0 & 0 \\
% & 0.035 & 50 & 0 & 0 & 0.0 & 0 & 0 & 0 & 0.0 & 0 \\
% & 0.03 & 65 & 1 & 0 & 0.3 & 6 & 0 & 0 & 0.0 & 0 \\
% & 0.025 & 82 & 1 & 0 & 0.05 & 1 & 0 & 0 & 0.0 & 0 \\
% & 0.02 & 121 & 1 & 0 & 0.4 & 8 & 0 & 0 & 0.0 & 0 \\
% & 0.015 & 189 & 1 & 0 & 0.45 & 9 & 1 & 0 & 0.55 & 11 \\
% & 0.0075 & 676 & 4 & 0 & 1.45 & 16 & 5 & 0 & 1.8 & 19 \\
% & 0.005 & 1618 & 6 & 1 & 2.75 & 20 & 8 & 0 & 3.75 & 19 \\
% & 0.002 & 1619 & 5 & 0 & 2.0 & 18 & 8 & 0 & 4.0 & 19 \\
% & 0.0015 & 1619 & 5 & 0 & 2.0 & 18 & 8 & 0 & 4.0 & 19 \\
% & 0.001 & 1619 & 5 & 0 & 2.0 & 18 & 8 & 0 & 4.0 & 19 \\
\midrule
\texttt{pumsb*} & 0.45 & 1913 & 5\% & 80\% \\
%\multirow{5}{*}{\texttt{pumsb*}} & 0.55 & 305 & 0 & 0 & 0.0 & 0 & 4 & 0 & 0.6 & 3 \\
% & 0.5 & 679 & 0 & 0 & 0.0 & 0 & 0 & 0 & 0.0 & 0 \\
% & 0.49 & 804 & 23 & 0 & 9.25 & 13 & 5 & 0 & 0.75 & 5 \\
% & 0.475 & 1050 & 0 & 0 & 0.0 & 0 & 1 & 0 & 0.28 & 6 \\
% & 0.45 & 1913 & 96 & 0 & 4.8 & 1 & 25 & 0 & 13.7 & 16 \\
% & 0.4 & 27354 & 59 & 0 & 16.9 & 9 & 18 & 0 & 3.6 & 8 \\
% & 0.35 & 116787 & 377 & 1 & 161.05 & 20 & 424 & 40 & 136.45 & 20 \\
% & 0.32 & 239691 & 615 & 1 & 136.0 & 20 & 381 & 104 & 169.2 & 20 \\
% & 0.3 & 432698 & 1081 & 1 & 320.85 & 20 & 2377 & 219 & 1190.05 & 20 \\
\midrule
\texttt{retail} & 0.0075 & 277 & 10\% & 20\% \\
%\multirow{6}{*}{\texttt{retail}} & 0.03 & 32 & 0 & 0 & 0.0 & 0 & 0 & 0 & 0.0 & 0 \\
% & 0.025 & 38 & 0 & 0 & 0.0 & 0 & 0 & 0 & 0.0 & 0 \\
% & 0.0225 & 46 & 0 & 0 & 0.0 & 0 & 0 & 0 & 0.0 & 0 \\
% & 0.02 & 55 & 0 & 0 & 0.0 & 0 & 0 & 0 & 0.0 & 0 \\
%% & 0.015 & 84 & 0 & 0 & 0.0 & 0 & 0 & 0 & 0.0 & 0 \\
% & 0.01 & 159 & 1 & 0 & 0.05 & 1 & 1 & 0 & 0.2 & 4 \\
% & 0.0075 & 277 & 2 & 0 & 0.5 & 9 & 2 & 0 & 0.25 & 4 \\
\bottomrule
\end{tabular}
\caption{Fractions of times that $\FI(\Ds,\Itm,\theta)$ contained false positives
and missed TFIs (false negatives) over 20 datasets from the same ground truth.}
\label{table:fp}
\end{table*}


%\subsection{Implementation.}\label{sec:implementation}
\subsection{Implementation.}
We implemented the algorithm in Python 3.3. %, except for the FIs
%mining algorithm and the optimization problem solver, as the methods are
%agnostic to these choices. 
To mine the FIs, we used the C implementation by Grahne
and Zhu~\citep{GrahneZ03}. %(written in C) available from the FIMI'03 implementation repository\footnote{\url{http://fimi.ua.ac.be/src/}}.
Our solver of choice for the SUKPs was IBM\textsuperscript{\textregistered}
ILOG\textsuperscript{\textregistered} CPLEX\textsuperscript{\textregistered}
Optimization Studio 12.3. %, called through the Python 2.7 API. 
We run the experiments on a number of machines with x86-64 processors running
GNU/Linux 3.2.0.

\subsection{Datasets generation.}\label{sec:dsgen}
We evaluated the algorithm using pseudo-artificial datasets generated by taking the datasets from the FIMI'04
repository\footnote{\url{http://fimi.ua.ac.be/data/}} as the \emph{ground truth} for the true frequencies
$\tfreq$ of the itemsets. We considered the following datasets:
\texttt{accidents}, %~\citep{GeurtsWBV03}, 
\texttt{BMS-POS}, \texttt{chess},
\texttt{connect}, \texttt{kosarak}, \texttt{pumsb\textsuperscript{*}}, and
\texttt{retail}. %~\citep{BrijsSVW99}. 
These datasets differ in size, number of
items, and, more importantly for our case, distribution of the frequencies of the
itemsets~\citep{GoethalsZ04}. We
created a dataset by \emph{sampling 20 million transactions uniformly at random} from a FIMI
repository dataset. In this way the 
the true frequency of an itemset is its frequency in the
original FIMI dataset. Given that our method to find the TFIs is
distribution-free, this is a valid procedure to establish a ground truth. We used
these enlarged datasets in our experiments, and use the original name of the
datasets in the FIMI repository to annotate the results for the datasets we generated.

\iffalse
\begin{table*}[tbp]
  \centering
  \begin{tabular}{llrcccccc}
    \toprule
    & & & \multicolumn{4}{c}{Reported TFIs (Average Fraction)} \\
    \cmidrule(l){4-7}
    & & & \multicolumn{2}{c}{``Vanilla'' (no info)} &
    \multicolumn{2}{c}{Additional Info} &
    \multicolumn{2}{c}{$\varepsilon_2\times 10^{6}$ (This Work)}\\
    \cmidrule(l){4-5} \cmidrule(l){6-7}  \cmidrule(l){8-9}
    Dataset & Freq.~$\theta$ & TFIs & CU Method & This Work & CU Method & This
    Work & ``Vanilla'' & Add.~Info\\
\midrule
\multirow{8}{*}{\texttt{accidents}} & 0.8 & 149 & 0.838& \bf 0.981
& 0.853& \bf 0.981 & 547 & 547\\
 & 0.7 & 529 & 0.925& \bf 0.985& 0.935& \bf 0.985 & 547 & 547\\
 & 0.6 & 2074 & 0.967& \bf 0.992& 0.973& \bf 0.992 & 569 & 569\\
 & 0.5 & 8057 & 0.946& \bf 0.991& 0.955& \bf 0.991 & 569 & 569\\
 & 0.45 & 16123 & 0.948& \bf 0.992& 0.955& \bf 0.992 & 590 & 599\\
 & 0.4 & 32528 & 0.949& 0.991& 0.957& \bf 0.992 & 611 & 611\\
 %& 0.35 & 68222 & 0.951297235 &  & 0.957897013 & 0.990999238 \\
 & 0.3 & 149545 & & & 0.957& \bf 0.989 & & 631\\
 %& 0.25 & 346525 & 0.952457687 &  & 0.958788688 & 0.989381718 \\
 & 0.2 & 889883 & &  & 0.957& \bf 0.987 & & 670\\
\midrule
\multirow{6}{*}{\texttt{BMS-POS}} & 0.05 & 59 & 0.845& \bf 0.938&
0.851& \bf 0.938 & 590 & 590\\
 & 0.03 & 134 & 0.879& \bf 0.992& 0.895& \bf 0.992 & 611 & 611 \\
 & 0.02 & 308 & 0.847& \bf 0.956& 0.876& \bf 0.956 & 631 & 631 \\
 & 0.01 & 1099 & 0.813& 0.868& 0.833& \bf 0.872 & 688 & 670\\
 & 0.0075 & 1896 & &  & 0.826& \bf 0.854 & & 670\\
 & 0.005 & 4240 & &  & 0.762& \bf 0.775 & & 631\\
\midrule
\multirow{5}{*}{\texttt{chess}} & 0.8 & 8227 & 0.964& \bf 0.991&
0.964& \bf 0.991 & 547 & 547\\
 & 0.775 & 13264 & 0.957 & \bf 0.990 & 0.957 & \bf 0.990 & 569 & 569\\
 & 0.75 & 20993 & 0.957& \bf 0.983& 0.957& \bf 0.983 & 569 & 569\\
 %& 0.7 & 48731 & 0.970328949 &  & 0.970328949 & 0.993779114 \\
 & 0.65 & 111239 & &  & 0.972& \bf 0.991 & & 611\\
 & 0.6 & 254944 & &  & 0.970& \bf 0.989 & & 631\\
\midrule
\multirow{5}{*}{\texttt{connect}} & 0.95 & 2201 & 0.802& \bf 0.951&
0.802& \bf 0.951 & 547 & 547 \\
 & 0.925 & 9015 & 0.881& \bf 0.975& 0.881& \bf 0.975 & 569 & 569\\
 & 0.9 & 27127 & 0.893& \bf 0.978& 0.893& \bf 0.978 & 591 & 591\\
 & 0.875 & 65959 & &  & 0.899& \bf 0.974 & & 611 \\
 & 0.85 & 142127 & &  & 0.918& \bf 0.974 & & 631\\
\midrule
\multirow{4}{*}{\texttt{kosarak}} & 0.04 & 42 & 0.738& \bf 0.939&
0.809& \bf 0.939 & 688 & 688\\
 & 0.035 & 50 & 0.720& \bf 0.980& 0.780& \bf 0.980 & 706 & 688\\
% & 0.03 & 65 & 0.646153846 &  & 0.738461538 & 0.923076923 \\
 & 0.025 & 82 & &  & 0.682& \bf 0.963 & & ''\\
 & 0.02 & 121 & &  & 0.650& \bf 0.975 & & ''\\
 & 0.015 & 189 & &  & 0.641& \bf 0.933 & & ''\\
\midrule
\multirow{5}{*}{\texttt{pumsb*}} & 0.55 & 305 & 0.791& \bf 0.926&
0.859& \bf 0.926 & 611 & 611\\
 & 0.5 & 679 & 0.929& \bf 0.998& 0.957& \bf 0.998 & '' & ''\\
 & 0.49 & 804 & 0.858& \bf 0.984& 0.907& \bf 0.984 & '' & ''\\
 & 0.475 & 1050 & &  & 0.942& \bf 0.996 & & ''\\
 & 0.45 & 1913 & &  & 0.861& \bf 0.976 & & ''\\
\midrule
\multirow{5}{*}{\texttt{retail}} & 0.03 & 32 & 0.625& \bf 1.00&
0.906& \bf 1.00 & 670 & 670\\
 & 0.025 & 38 & 0.842& \bf 0.973& 0.972& \bf 0.973 & 688 & ''\\
 & 0.0225 & 46 & 0.739& 0.934& 0.869& \bf 0.935 & 724 & ''\\
 & 0.02 & 55 & &  & 0.882& \bf 0.945 & & ''\\
 & 0.01 & 159 & &  & 0.902& \bf 0.931 & & ''\\
 & 0.0075 & 277 & &  & 0.811& \bf 0.843 & & ''\\
 \bottomrule
 \end{tabular}
  \caption{Recall. Average fraction (over 20 runs) of reported TFIs
  in the output of an algorithm using Chernoff and Union bound and of the one
  presented in this work. For each algorithm we present two versions, one
  (Vanilla) which uses no information about the generative process, and one
  (Add.~Info) in which we assume the knowlegde that the process will not
  generate any transaction longer than twice the size of the longest transaction
  in the original FIMI dataset. In bold, the best result (highest reported
  fraction). The two rightmost columns report the value of
  $\varepsilon_2$ computed by our algorithm in the two cases.}
\label{table:power}
\end{table*}
\fi

\begin{table*}[tbp]
  \centering
  \begin{tabular}{llrcccc}
    \toprule
    & & & \multicolumn{4}{c}{Reported TFIs (Average Fraction)} \\
    \cmidrule(l){4-7}
    & & & \multicolumn{2}{c}{``Vanilla'' (no info)} &
    \multicolumn{2}{c}{Additional Info}\\
    \cmidrule(l){4-5} \cmidrule(l){6-7}
    Dataset & Freq.~$\theta$ & TFIs & CU Method & This Work & CU Method & This Work\\
\midrule
\multirow{8}{*}{\texttt{accidents}} & 0.8 & 149 & 0.838& \bf 0.981
& 0.853& \bf 0.981\\
 & 0.7 & 529 & 0.925& \bf 0.985& 0.935& \bf 0.985\\
 & 0.6 & 2074 & 0.967& \bf 0.992& 0.973& \bf 0.992\\
 & 0.5 & 8057 & 0.946& \bf 0.991& 0.955& \bf 0.991\\
 & 0.45 & 16123 & 0.948& \bf 0.992& 0.955& \bf 0.992\\
 & 0.4 & 32528 & 0.949& 0.991& 0.957& \bf 0.992\\
 %& 0.35 & 68222 & 0.951297235 &  & 0.957897013 & 0.990999238 \\
 & 0.3 & 149545 & & & 0.957& \bf 0.989\\
 %& 0.25 & 346525 & 0.952457687 &  & 0.958788688 & 0.989381718 \\
 & 0.2 & 889883 & &  & 0.957& \bf 0.987\\
\midrule
\multirow{6}{*}{\texttt{BMS-POS}} & 0.05 & 59 & 0.845& \bf 0.938&
0.851& \bf 0.938\\
 & 0.03 & 134 & 0.879& \bf 0.992& 0.895& \bf 0.992\\
 & 0.02 & 308 & 0.847& \bf 0.956& 0.876& \bf 0.956\\
 & 0.01 & 1099 & 0.813& 0.868& 0.833& \bf 0.872\\
 & 0.0075 & 1896 & &  & 0.826& \bf 0.854\\
 & 0.005 & 4240 & &  & 0.762& \bf 0.775\\
\midrule
\multirow{5}{*}{\texttt{chess}} & 0.8 & 8227 & 0.964& \bf 0.991&
0.964& \bf 0.991\\
 & 0.775 & 13264 & 0.957 & \bf 0.990 & 0.957 & \bf 0.990 \\
 & 0.75 & 20993 & 0.957& \bf 0.983& 0.957& \bf 0.983\\
 %& 0.7 & 48731 & 0.970328949 &  & 0.970328949 & 0.993779114 \\
 & 0.65 & 111239 & &  & 0.972& \bf 0.991\\
 & 0.6 & 254944 & &  & 0.970& \bf 0.989\\
\midrule
\multirow{5}{*}{\texttt{connect}} & 0.95 & 2201 & 0.802& \bf 0.951&
0.802& \bf 0.951\\
 & 0.925 & 9015 & 0.881& \bf 0.975& 0.881& \bf 0.975 \\
 & 0.9 & 27127 & 0.893& \bf 0.978& 0.893& \bf 0.978\\
 & 0.875 & 65959 & &  & 0.899& \bf 0.974\\
 & 0.85 & 142127 & &  & 0.918& \bf 0.974\\
\midrule
\multirow{4}{*}{\texttt{kosarak}} & 0.04 & 42 & 0.738& \bf 0.939&
0.809& \bf 0.939\\
 & 0.035 & 50 & 0.720& \bf 0.980& 0.780& \bf 0.980\\
% & 0.03 & 65 & 0.646153846 &  & 0.738461538 & 0.923076923 \\
 & 0.025 & 82 & &  & 0.682& \bf0.963\\
 & 0.02 & 121 & &  & 0.650& \bf 0.975\\
 & 0.015 & 189 & &  & 0.641& \bf 0.933\\
\midrule
\multirow{5}{*}{\texttt{pumsb*}} & 0.55 & 305 & 0.791& \bf 0.926&
0.859& \bf 0.926\\
 & 0.5 & 679 & 0.929& \bf 0.998& 0.957& \bf 0.998\\
 & 0.49 & 804 & 0.858& \bf 0.984& 0.907& \bf 0.984\\
 & 0.475 & 1050 & &  & 0.942& \bf 0.996\\
 & 0.45 & 1913 & &  & 0.861& \bf 0.976\\
\midrule
\multirow{5}{*}{\texttt{retail}} & 0.03 & 32 & 0.625& \bf 1.00&
0.906& \bf 1.00\\
 & 0.025 & 38 & 0.842& \bf 0.973& 0.972& \bf 0.973\\
 & 0.0225 & 46 & 0.739& 0.934& 0.869& \bf 0.935\\
 & 0.02 & 55 & &  & 0.882& \bf 0.945\\
 & 0.01 & 159 & &  & 0.902& \bf 0.931\\
 & 0.0075 & 277 & &  & 0.811& \bf 0.843\\
 \bottomrule
 \end{tabular}
  \caption{Recall. Average fraction (over 20 runs) of reported TFIs
  in the output of an algorithm using Chernoff and Union bound and of the one
  presented in this work. For each algorithm we present two versions, one
  (Vanilla) which uses no information about the generative process, and one
  (Add.~Info) in which we assume the knowlegde that the process will not
  generate any transaction longer than twice the size of the longest transaction
  in the original FIMI dataset. In bold, the best result (highest reported
  fraction).}
\label{table:power}
\end{table*}
%\begin{table*}[tb]
%\centering
%\begin{tabular}{llrcccccc}
%\toprule
%  & & & & & & & \multicolumn{2}{c}{Reported TFIs (\%)} \\
%  \cmidrule(l){8-9}
%%& & & \multicolumn{4}{c}{Reported TFIs(\%)} \\
%%\cmidrule(l){4-7}
%%& & & \multicolumn{2}{c}{Split dataset} & \multicolumn{2}{c}{Full dataset} \\
%%\cmidrule(l){4-5} 
%%\cmidrule(l){6-7} 
%%Dataset & $\theta$ & TFIs & Holdout & Method 1 & Bonferroni & Method 2\\
%Dataset & $\theta$ & TFIs & Avg FPs & Times FPs & Avg FNs & Times FNs & Bonferonni
%& Method 1 \\
%\midrule
%\multirow{6}{*}{\texttt{accidents}} & 
%0.8 & 149 & 0.0 & 0 & 0.0 & 0 & 83.893 & \bf 95.974 \\
%& 0.7 & 529 & 0.1 & 2 & 0.7 & 7 & 92.439&\bf 98.488\\
%& 0.6 & 2074 & 1.75 & 14 & 0.5 & 7& 96.625&\bf 99.132 \\
%& 0.5 & 8057 & 8.05 & 19 & 3.65 & 17 & 94.551&\bf 99.081\\
%& 0.45 & 16123 & 10.1 & 20 & 6.35 & 20 & 94.691&\bf 99.076\\
%& 0.4 & 32528 & 26.55 & 20 & 16.55 & 20 & 94.774&\bf 98.970\\
%%0.8& 149&  94.631& \bf 97.987& 83.893 & \bf 95.973\\
%%& 0.7& 	529&    97.543&\bf 98.488 & 92.439&\bf 98.488 \\
%%& 0.6& 	2074&   98.505&\bf 98.987& 96.625&\bf 99.132 \\
%%& 0.5& 	8057&   98.349&\bf 99.007& 94.551&\bf 99.081\\
%%& 0.45& 16123&  98.177&\bf 98.915& 94.691&\bf 99.076\\
%%& 0.4& 	32528&  98.032&\bf 98.761& 94.774&\bf 98.970\\
%%& 0.35&	68222&  98.140& \bf 98.666& \\
%%& 0.3& 	149545& 98.033& \bf 98.529& \\
%%& 0.25&	346525& 98.165& \bf 98.382& \\
%%& 0.2& 	889883& 97.995& \bf 98.057& \\
%\midrule
%\multirow{4}{*}{\texttt{BMS-POS}} & 
%0.05 & 59 & 0.0 & 0 & 0.3 & 6 & 84.746&\bf 93.220\\
%& 0.03 & 134 & 0.05 & 1 & 0.0 & 0 & 88.060&\bf 99.254\\
%& 0.02 & 308 & 0.85 & 14 & 0.2 & 3 & 84.740&\bf 95.455\\
%& 0.01 & 1099 & 1.45 & 20 & 3.85 & 20 & 81.620&\bf 88.171\\
%%0.05& 59&\bf 98.305& 93.220& 84.746&\bf 93.220\\
%%& 0.03&	   134&\bf 99.254&\bf 99.254& 88.060&\bf 99.254\\
%%& 0.02&	   308&\bf 98.377& 95.130& 84.740&\bf 95.455\\
%%& 0.01&	  1099&\bf 95.814& 85.805& 81.620&\bf 88.171\\
%%& 0.0075& 1896&\bf 96.097& 82.331& & \\
%%%& 0.005&  4240&\bf 94.410& 72.004& & \\
%\midrule
%\multirow{2}{*}{\texttt{chess}} & 
%0.8 & 8227 & 0.0 & 0 & 7.45 & 17& 96.475&\bf 99.198\\
%& 0.75 & 20993 & 0.0 & 0 & 68.25 & 20 & 95.756&\bf 98.418\\
%%0.8& 8227& 97.265&\bf 98.918& 96.475&\bf 99.198\\
%%& 0.75&	20993&   96.375&\bf 98.042& 95.756&\bf 98.418\\
%%& 0.7&	48731&   97.613&\bf 99.001& \\
%%& 0.6&	254944&  97.243&\bf 98.464& \\
%%& 0.5&	1272932& 97.352&\bf 98.293& \\
%\midrule
%%\multirow{3}{*}{\texttt{connect}} & 0.9& 27127& 93.217&\bf 97.213& 89.369&\bf 97.921\\
%\texttt{connect} & 
%0.9 & 27127 & 12.8 & 10 & 28.6 & 16 & 89.369 & \bf 97.921\\
%%0.9& 27127& 93.217&\bf 97.213& 89.369&\bf 97.921\\
%%& 0.875 & 65959& 94.481& \bf 96.040& \\
%%& 0.85&	142127& 94.242&\bf 96.882& \\
%%& 0.8&	533975& 95.047&\bf 97.464& \\
%\midrule
%\multirow{2}{*}{\texttt{kosarak}} & 
%0.04 & 42 & 0.0 & 0 & 0.0 & 0 & 73.810&\bf 95.238\\
%& 0.035 & 50 & 0.0 & 0 & 0.0 & 0 & 72.000&\bf 98.000\\
%%0.04& 42&\bf 97.620&\bf 97.620& 73.810&\bf 95.238\\
%%& 0.035&   50&\bf 100&    98.000& 72.000&\bf 98.000\\
%%& 0.03&	   65&\bf 96.923& 93.846& \\
%%& 0.025&   82&\bf 98.780& 97.561& \\
%%& 0.02&	  121&\bf 100&    97.521& \\
%%& 0.015&  189&\bf 98.942& 93.122& \\
%%& 0.01&	  383&\bf 96.606& 86.945& \\
%%& 0.005& 1618&\bf 96.415& 74.351& \\
%\midrule
%\multirow{3}{*}{\texttt{pumsb\textsuperscript{*}}} & 
%0.55 & 305 & 0.0 & 0 & 0.6 & 3 & 80.984&\bf 95.401\\
%& 0.5 & 679 & 0.0 & 0 & 0.0 & 0 & 92.931&\bf 99.853 \\
%& 0.49 & 804 & 9.25 & 13 & 0.75 & 5 & 86.692&\bf 98.756\\
%%0.55& 305& 92.131&\bf 94.426& 80.984&\bf 95.401\\
%%& 0.5& 	679&    \bf 99.853&\bf  99.853& 92.931&\bf 99.853 \\
%%& 0.49&	804&    97.637&\bf 98.507& 86.692&\bf 98.756\\
%%& 0.45&	1913&   96.968&\bf  97.909& \\
%%& 0.4& 	27354&  98.183&\bf  99.101& \\
%%& 0.35&	116787& 96.435&\bf  96.972& \\
%%& 0.3& 	432698& 96.567&\bf  97.326& \\
%\midrule
%\multirow{3}{*}{\texttt{retail}} & 
%0.03 & 32 & 0.0 & 0 & 0.0 & 0 & 62.500&\bf  100\\
%& 0.025 & 38 & 0.0 & 0 & 0.0 & 0 & 84.211&\bf 97.368\\
%& 0.02 & 55 & 0.0 & 0 & 0.0 & 0 & 72.340&\bf 95.745\\
%%0.03& 32&\bf  100&\bf  100& 62.500&\bf  100\\
%%& 0.025& 38&\bf   100&    97.368& 84.211&\bf 97.368\\
%%& 0.02&	 55&\bf   100&    94.545& 72.340&\bf 95.745\\
%%& 0.015& 84&\bf   100&    95.238& \\
%%& 0.01&	 159&\bf  99.371& 93.082& \\
%%& 0.005& 580&\bf  96.552& 79.655& \\
%\bottomrule
%\end{tabular}
%\caption{Statistical power of various methods to extract the True Frequent
%Itemsets while controlling the FWER ($\delta=0.1$). In bold the best result for
%each setting (split or full dataset).}
%\label{table:power}
%\end{table*}

\subsection{False positives and false negatives in
$\FI(\Ds,\Itm,\theta)$.}
%\subsection{Existence of false positives and false negatives}
In the first set of experiments we evaluated the performances, in terms of
inclusion of false positives and false negatives in the output, of mining the
dataset at frequency $\theta$. Table~\ref{table:fp} reports  
the fraction of times (over 20 datasets from the same
ground truth) that the set $\FI(\Ds,\Itm,\theta)$ contained false positives
(FP) and was missing TFIs (false negatives (FN)). In most cases, especially when there are
many TFIs, the inclusion of false positives when mining at frequency $\theta$
should be expected. This highlights a need for methods like the one presented in
this work, as there is no guarantee that $\FI(\Ds,\Itm,\theta)$
only contains TFIs. On the other hand,
the fact that some TFIs have frequency in the dataset \emph{smaller} than
$\theta$ (false negatives) points out how one can not aim to extract all and
only the TFIs by using a fixed threshold approach (as the one we present).

\subsection{Control of the false positives (Precision).}\label{sec:fwer}
%We evaluated the capability of our methods to control the FWER by first creating
%a number of enlarged datasets, %with 20 million transactions
%%each as described in Section~\ref{sec:dsgen}
%and then run our methods to extract a collection of TFIs from thee dataset.
In this set of experiments we evaluated how well the threshold $\hat\theta$
computed by our algorithm allows to avoid the inclusion of false negatives in
$\FI(\Ds,\Itm,\hat\theta)$. %, i.e., itemsets with true frequency lower than the minimum frequency threshold.
To this end, we used a wide range of values for the minimum true frequency
threshold $\theta$ (see Table~\ref{table:power}) and fixed $\delta=0.1$. We
repeated each experiment on 20 different enlarged datasets generated from the
same original FIMI dataset. %distribution, for each original distribution. 
In all the \emph{hundreds} of runs of our algorithms, %the returned collection of itemsets 
$\FI(\Ds,\Itm,\hat\theta)$ \emph{never} contained \emph{any false
positive}, i.e., \emph{always contained only TFIs}. In other words, the
\emph{precision} of the output was 1.0 in all our experiments. Not %This means that not 
only our
method can give a frequency threshold to extract only TFIs, but %that 
it is more
\emph{conservative}, in terms of including false positives, than what the
theoretical analysis guarantees. %is guaranteed by the theoretical analysis.%, since
%the returned collection
%never contained any false positive even if the upper bound $\delta$ to the
%probability of such event was set to $0.1$. This is expected as the
%(empirical) b-index is not always a tight bound to the (empirical) VC-dimension.

%\subsection{Statistical power and comparison with other methods}
%\subsection{Power of the algorithm}
\subsection{Inclusion of TFIs (Recall).}
%The power of a statistical test is the probability that the test will reject the
%null hypothesis when the null hypothesis is false. In the case of TFIs, this
%corresponds to the probability of including a TFI in the output collection. 
%It
%is often difficult to
%analytically quantify the statistical power of a test, especially
%in multiple hypothesis testing settings with correlated hypotheses.
%This is indeed our case, and we therefore conducted an empirical evaluation of
%the statistical power of our methods by
In addition to avoid false positives in the results, one wants to include as many TFIs
as possible in the output collection. To this end, we assessed what fraction of
the total number of TFIs is reported in $\FI(\Ds,\Itm,\hat\theta)$. Since there
were no false positives, this is corresponds to evaluating the \emph{recall} of
the output collection. We fixed $\delta=0.1$, and considered different values
for the minimum true frequency threshold $\theta$ (see Table~\ref{table:power}).
For each frequency threshold, we repeated the experiment on 20 different
datasets sampled from the same original FIMI dataset, %on different datasets,
%generated as described in Sect.~\ref{sec:dsgen}
and found very small variance in the results.  %In order to evaluate the performances of our algorithm, 
We compared the fraction of TFIs that our algorithm included in output with that
included by the ``Chernoff and Union bounds'' (CU) method we presented in
Introduction. We compared two variants of the algorithms: one
(``vanilla'') which makes no assumption on the generative distribution $\prob$,
and another (``additional info'') which assumes that the process will not
generate any transaction longer than twice the longest transaction found in the
original FIMI dataset. Both algorithms can be easily modified to include this
information. In Table~\ref{table:power} we report the average fraction of TFIs
contained in $\FI(\Ds,\Itm,\hat\theta)$. 
%\textcolor{Red}{We also report the
%values of $\varepsilon_2$ used to compute $\hat\theta=\theta+\varepsilon_2$ in
%our method (see Sect.~\ref{sec:main})}. 
We can see that the amount of TFIs
found by our algorithm is always very high: only a \emph{minimal}
fraction (often less than $3\%$) %number 
of TFIs do not appear in the output. This is explained by the fact that the
value $\varepsilon_2$ computed in our method (see Sect.~\ref{sec:main}) is
always smaller than $10^{-4}$. Moreover our solution
\emph{uniformly outperforms} the CU method, often by a huge margin, since
 our algorithm does not have to take into account all
possible itemsets when computing $\hat\theta$. % 
%The fraction of reported TFIs was always very high (Table~\ref{table:power})
%In the two rightmost columns of Table~\ref{table:power} we can see the results
%for two variants of our algorithm: in the ``Vanilla'' variant, we assume no
%additional knowledge about the unknown generative distribution $\prob$, while in
%the ``Additional info'' variant we assume to know that $\prob$ will not generate
%transactions longer than twice the longest transaction in the original dataset.
%We can see that the recall of our method is very high in general in both cases,
%but the inclusion of additional knowledge improves the practical usefulness of
%our algorithm and, in some cases, allows to achieve higher recall. 
Only partial results are reported for the ``vanilla'' variant because of the
very high number of items in the considered datasets: the mining of the
dataset is performed at frequency threshold $\theta-\varepsilon_1$  and if there
are many items, then the value of $\varepsilon_1$ becomes very high because the
bound to the VC-dimension of $\range(2^\Itm)$ is $|\Itm|-1$, and as a
consequence we have $\theta-\varepsilon_1\le 0$. %which drives the value of
%$\varepsilon_1$ very high, and limits the applicability of our method. 
We stress, though, that assuming no knowledge about the distribution $\prob$ is not realistic, and
usually additional information, especially regarding the length of the
transactions, is available and can and should be used. The use of 
additional information gives flexibility to our method and improves its
practicality. Moreover, in some cases, it allows to find an even larger fraction
of the TFIs.

%We also wanted to compare the performances of our algorithm %methods 
%with that of a baseline method to control false positives. %established techniques to identify TFI's.
%can control the probability of Type-1
%errors. 
%To
%this end we adapted the holdout technique proposed by~\citet{Webb07} to the TFIs
%problem and compared its power with that of Method 1.
%The goal of the holdout method is to reduce the search space
%with the goal of mitigating the correction necessary for the multiple hypothesis
%testing.
%In this technique, the dataset is randomly split into two portions, an
%\emph{exploratory} part and an \emph{evaluation} part. First the frequent
%itemsets w.r.t.~$\theta$ are extracted from the exploratory part. Then,
%each of the null hypotheses of these itemsets is tested using a Binomial test on
%the frequency of the itemsets in the evaluation dataset, with critical value
%$\delta/k$, where $k$ is the number of patterns found in the first step. 
%We implemented the holdout method using the Bonferroni
%correction and compared the power of this method with that of our statistical
%test. 
%To evaluate the power of Method 2, which does not need to split the
%dataset, we compared its results with the power of a method that uses the
%In Bonferroni correction over all possible itemsets to test, with a Binomial test,
%the FIs w.r.t.~$\theta$ obtained from the entire dataset.

%The statistical power evaluation and the comparison with the
%The average (over 20 runs) fraction of TFI's reported by the two methods are
%presented in Table~\ref{table:power}, in the three rightmost columns.
%holdout technique are presented in Table~\ref{table:power}, where we reported
%the average over 20 runs for each experiment. 
%We highlighted in bold the
%best result for each experiment and each of the two setting (split or full
%dataset).
%Only partial results are reported
%for the full dataset approaches because of issues with the amount of
%computational resources required by this method when there is a very high
%number of itemsets in the negative border.
%It is possible to appreciate that
%The power of our algorithm is very high in general, for all the tested datasets
%and values for $\theta$. We can notice
%The comparison of Method 1 with the
%holdout technique shows that the relative performances of these methods are
%dataset specific: the statistical power seems to depend heavily on the
%true frequency distribution of the itemsets. From the table it is possible to
%see that Method 1 performs better than the holdout technique when the number
%of TFIs is high, while the opposite is true when the minimum threshold frequency
%$\theta$ is very low. We conjecture that this is due to the fact that the value
%for $\theta$ is not taken in consideration when computing the $\varepsilon$ to
%define the acceptance region $[0,\theta+\varepsilon]$ in Method 1, as the
%computation of $\varepsilon$ in Thms.~\ref{thm:eapprox}
%and~\ref{thm:eapproxempir} does not depend on $\theta$. In future work, we plan to
%investigate whether it is possible to include $\theta$ as part of the
%computation of $\varepsilon$. On the other hand the direct correction applied
%in the holdout technique becomes less and less powerful as the number of TFIs
%increases because it does not take into account the many correlations between the
%itemsets. Given that in common applications the number of frequent itemsets of
%interest is large, we believe that in real scenarios Method 1 can be more
%effective in extracting the True Frequent Itemsets while rigorously controlling
%the probability of false discoveries. 
%One can see that our algorithm performs much better than the %For the full dataset case, we can see
%that Method 2 performs much better than the 
%CU (Chernoff+Union bounds) method in all cases, thanks to the fact that it does
%not need to take into consideration all possible itemsets when computing the
%threshold $\hat\theta$. %when computing the acceptance region, all possible hypotheses. 
%One can also compare Method 1 and Method 2 to each other, given that Method 2 can be used in cases where Method 1
%can. Method 2 manages to achieve higher statistical power than Method 1 in
%almost all cases, but we must stress that this is at the expense of a longer
%running time and an higher amount of required memory.

%\subsection{Runtime evaluation}
%We found our methods to be quite efficient in practice in terms of time needed
%to compute the output collection. The running time is dominated by the time
%needed to compute the negative border (only for Method 2) and to solve the
%SUKPs. Various optimizations and shortcuts can be implemented to substantially
%speed up our methods (see also the discussion in Sect.~\ref{sec:range}). Due
%to space restrictions, we defer the presentation of the runtime evaluation to
%the full version of the paper. 

%\subsection{Estimation of the True Frequency}\label{sec:freqest}
%\XXX I'd put this only if we actually have the space. \MR
%
%Recalling the definition of an $\varepsilon$-approximation to a range set
%$R$ (Def.~\ref{def:eapprox}) one may ask how close the frequency of
%an RFI in the sample is to its real frequency in the distribution. It is
%important to note, though, that the error bound expressed in the definition of
%an $\varepsilon$-approximation is only valid for the members of $R$. In the case
%of the RFI's, the itemsets $A$ such that $T(A)\in
%R_{\mathcal{B}^-(\RFI(p,\Itm,\theta))}$ are \emph{not} RFI's. Therefore, provided
%the dataset is an $\varepsilon''$-approximation, none of them will be included
%in the output collection of itemsets computed by our method. This implies that
%our method can not guaranteed any bound on the error $|f_\Ds(X)-r_p(X)|$ for the
%itemsets included in the output. Nevertheless, the frequencies in the sample are
%very close to the real frequencies, orders of magnitude closer than
%$\varepsilon$. In Table~\ref{table:freqapprox} we report the maximum and the
%average error $|f_\Ds(X)-r_p(X)|$ for the same datasets we used in the
%evaluation of the statistical power of our method, with $\delta=0.1$.
%%
%\begin{table}[tb]
%  \centering
%% \subfloat[Absolute frequency error]{\label{table:freqapprox}
%\begin{tabular}{lccc}
%\toprule
%  & & \multicolumn{2}{c}{Real frequency estimation error} \\
% \cmidrule(l){3-4}
%Dataset  & $\theta$ & maximum & average \\
%\midrule
%\texttt{accidents} & 0.4 & $4.3\cdot10^{-4}$ &  $9.2\cdot10^{-5}$ \\
%\texttt{BMS-POS} & 0.01 & $2.1\cdot10^{-4}$ & $2.4\cdot10^{-5}$ \\
%\texttt{kosarak} & 0.04 & $2.1\cdot10^{-4}$ & $4.6\cdot10^{-5}$ \\
%\texttt{pumsb\textsuperscript{*}} & 0.45 & $3.7\cdot10^{-4}$ & $9.7\cdot10^{-5}$ \\
%\texttt{retail} &  0.022 & $2.3\cdot10^{-4}$ & $4.0\cdot 10^{-5}$ \\
%\bottomrule
%\end{tabular}
%}
%\caption{Real frequency estimation error}
%\label{table:freqapprox}
%\end{table}

