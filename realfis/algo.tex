\section{Finding the True Frequent Itemsets}\label{sec:main}
In this section we present an algorithm to identify a threshold $\hat{\theta}$ such
that, with probability at least $\delta$ for some user-specified parameter
$\delta\in(0,1)$, all itemsets with frequency at least $\hat{\theta}$ in $\Ds$
are True Frequent Itemsets with respect to a fixed minimum true frequency
threshold $\theta\in(0,1]$. %In other words, we present an algorithm 
The threshold $\hat{\theta}$ can be used to find a collection
$\mathcal{C}=\FI(\Ds,\Itm,\hat\theta)$ of itemsets such that $\Pr(\exists
A\in\mathcal{C} \mbox{ s.t.~} \tfreq(A)<\theta)<\delta$.
%so that each part is still
%a collection of i.i.d.~samples from the data-generating distribution $\prob$. On
%the other hand, splitting the dataset may not always be possible \XXX: Why?\MR
%and we developed a second test for TFIs to handle this situation.

%The intuition behind our method is as follows. 
%MATTEO PLEASE ADD INTUITION
\iffalse
It starts by building a set
$\mathcal{G}$ of ``candidates TFIs''. For each itemset
$A\in\mathcal{G}$ we evaluate the probability that $\tfreq(A)$ is less than $\theta$
by considering the probability that  $f(A)$ of $A$ in
the dataset is observed when $\tfreq(A) < \theta$. If
the frequency falls into the \emph{acceptance region} %$\Delta=[0,\theta+\varepsilon)$
$[0,\theta+\varepsilon)$, where $\varepsilon$ is a function of $\delta$ and
$\mathcal{G}$ computed by our methods,
then $H_A$ is \emph{accepted}, otherwise $H_A$ is \emph{rejected}
and $A$ is flagged as True Frequent and included in the output collection
$\mathcal{C}$. Any
itemset not in $\mathcal{G}$ is not considered and will not
be reported in output. It should be clear that the definition of the acceptance
region is critical for the method to have the desired FWER at most
$\delta$: one needs to compute an $\varepsilon$ such that 
\[ 
\Pr\left(\exists A\in\mathcal{C}\mbox{ s.t. } \tfreq(A)<\theta\right)=\Pr\left(\exists A\in\mathcal{G} \mbox{ s.t. } \tfreq(A)<\theta \mbox{ and }
f(A)\ge\theta+\varepsilon\right)\le\delta\enspace.\]
\fi


%The two methods we present differ in the definition of $\mathcal{G}$ and in the
%computation of $\varepsilon$, but both use the tools we developed in
%Sect.~\ref{sec:range}.

