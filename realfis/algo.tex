\section{A Statistical Test for RFI's}\label{sec:main}
In this section we present our algorithm to mine a collection of RFI's
while guaranteeing that the FWER is within the user-specified parameter
$\delta$. We use results from statistical learning theory to achieve this goal.
We define two range sets made of subsets of transactions associated to two different
collections of itemsets, and we give upper bounds to their (empirical)
VC-dimensions. In Section~\ref{sec:algo}, we
present our algorithm to mine a collection of RFI's with FWER at most $\delta$.

\subsection{The range set of itemsets}\label{sec:rangeitemsets}
Given a ground set $\Itm$ of items, let $p$ be a probability distribution on $2^\Itm$,
and let $R_\Itm=\{T(A); A\in 2^{\Itm}\}$ be a range set
on $2^\Itm$. We call $R_\Itm$ the \emph{range set of the itemsets}. Given any
itemset $A$, it is easy to see that $\nu(T(A))=r_p(A)$. Let $\Ds$ be a dataset,
seen as a collection of i.i.d.~samples from the probability distribution $p$
defined on the transactions built on $\Itm$. It should be evident that, for any
itemset $A$, we have $\nu_\Ds(A)=f_\Ds(A)$.  

\iffalse
{\bf XXX:} In the reference for \citep{RiondatoU12} mention the extended revised
version from arXiv, from which the definition and the theorem below are taken
(but the numbers are the same from the PKDD one). We could just reference the
one from arXiv instead, but I prefer to reference the one from PKDD with the
additional specification, to give more ``weight.
\fi

\citet{RiondatoU12} gave an upper bound to the empirical VC-dimension of
$R_\Itm$ on $\Ds$. Before presenting the bound, we need to introduce a
characteristic quantity of the dataset $\Ds$ called the \emph{d-index} and
denoted as $\mathsf{d}(\Ds)$.
\begin{definition}\label{def:dindex}
  \emph{(\citep[Def.~12]{RiondatoU12})} Let $\Ds$ be a dataset. The
  \emph{d-index} $\mathsf{d}(\Ds)$ of $\Ds$ is the maximum integer $d$ such that
  $\Ds$ contains at least $d$ transactions of length at least $d$ such that for
  any two of these transactions $\tau',\tau''$ we have neither $\tau'\subseteq
  \tau''$ nor $\tau''\subseteq \tau'$.
\end{definition}

\citet{RiondatoU12} presents an efficient algorithm to compute an upper bound to
the d-index of a dataset.

\begin{theorem}\label{thm:empvcdimubfirst}
  \emph{(\citep[Thm.~3]{RiondatoU12})} Let $\Ds$ be a dataset  with
  transactions built on a ground set $\Itm$. Then $\EVC(R_\Itm,\Ds)\le
  \mathsf{d}(\Ds)$.
\end{theorem}

It easy to extend this theorem to prove an upper bound to the VC-dimension of
$R_\Itm$.
\begin{theorem}\label{thm:vcdimubfirst}
  Let $\Ds$ be a dataset with transactions built on a ground set $\Itm$. Then
  $\VC(R_\Itm)\le |\Itm|-1$.
\end{theorem}
\begin{proof}
  Let $k=|\Itm|$. Assume w.l.o.g.~that $\Itm=\{1,\dotsc,k\}$. Let
  $\tau_i=\Itm\setminus\{i\}$, $1\le i\le k$. We have $|\tau_i|=k-1$, for $1\le
  i\le k$. Notice that for $i\neq j$, neither $\tau_i\subseteq\tau_j$ nor
  $\tau_j\subseteq\tau_i$. Consider a dataset $\Ds$ made of the set of
  transactions $\{\tau_1,\dotsc,\tau_k\}$ and any number of additional
  transactions. It is easy to see that $\mathsf{d}(\Ds)=k-1$, and that
  it is impossible to build a dataset with a larger d-index using the items of
  $\Itm$, because there is no way to use the items of $\Itm$ to build $\ell>k-1$
  transactions of length at least $\ell$ and such that for any two of these
  transactions $\tau',\tau''$ we have neither $\tau'\subseteq \tau''$ nor
  $\tau''\subseteq \tau'$. This concludes the proof.
\end{proof}

\citet{RiondatoU12} also showed that the upper bound presented in
Thm.~\ref{thm:empvcdimubfirst} is strict, in the sense that there exists
datasets for which $\EVC(R_\Itm,\Ds)=\mathsf{d}(\Ds)$. This implies that the
upper bound presented in Thm.~\ref{thm:vcdimubfirst} is also strict.

The upper bound to the empirical VC-Dimension of $R_\Itm$ on $\Ds$ is used
by~\citet{RiondatoU12} to compute the size of a sample $\Sam$ needed to extract
a superset of the collection of $\FI(\Ds,\Itm,\theta)$ from $\Sam$, where the
quality of the approximation is controlled by user-defined parameters
$\varepsilon$ and $\delta$. In this work we will use the two upper bounds
presented above to compute a superset of the negative border of the collection
of RFI's with respect to a minimum real frequency threshold $\theta$.

\subsection{The range set of the negative border}\label{sec:rangenegbord}
We now define another range set that we will call \emph{the range set of the
negative border}. Given a minimum real frequency
threshold $\theta$, let
$\mathcal{B}^-_\theta=\mathcal{B}^-(\RFI(p,\Itm,\theta))$, the negative border
of the RFI's with respect to theta. We define the range set of the negative
border as
\[R_{\mathcal{B}^-_\theta}=\{T(A); A\in\mathcal{B}^-_\theta\}.\] 
Similar to the case of the range set of the itemsets, we have, for an itemset
$A\in\mathcal{B}^-_\theta$, $\nu(T(A))=r_p(A)$ and, for a dataset $\Ds$,
$\nu_\Ds(T(A))=f_\Ds(A)$.

The intuition behind the definition of this range space is the following.
Assume that we can compute an upper bound to the (empirical) VC-dimension of
this range space. Then, given a dataset $\Ds$ we can apply Theorem~\ref{thm:eapprox}
or Theorem~\ref{thm:eapproxempir} to compute an \emph{offset
parameter} $\varepsilon$ such that $\Ds$ is a $\varepsilon$-approximation to
$R_{\mathcal{B}^-_\theta}$ with probability at least $1-\delta$ for some
user-specified parameter $\delta$. Now, if $\Ds$ is indeed a
$\varepsilon$-approximation to $R_{\mathcal{B}^-_\theta}$, then no itemset
$A\in\mathcal{B}^-_\theta$ can have a frequency $f_\Ds(A)$ in $\Ds$ greater than
$\theta+\varepsilon$. We will prove this formally in Sec.~\ref{sec:algo} and
use this fact as the core step in the analysis of the correctness of our method
to compute a collection of RFI's with FWER at most $\delta$.

\subsubsection{An upper bound to the VC-Dimension of
$R_{\mathcal{B}^-_\theta}$}\label{sec:bindexub}
It is easy to see that $R_{\mathcal{B}^-_\theta}$ is a subset of $R_\Itm$ for
any $\theta\in[0,1]$, so we will have
$\EVC(R_{\mathcal{B}^-_\theta},\Ds)\le\EVC(R_\Itm,\Ds)$ for any $\Ds$ and
consequently, $\VC(R_{\mathcal{B}^-_\theta})\le\VC(R_\Itm)$. This means that one
could apply Theorems~\ref{thm:empvcdimubfirst}~and~\ref{thm:vcdimubfirst} to
bound $\EVC(R_{\mathcal{B}^-_\theta},\Ds)$ and $\VC(R_{\mathcal{B}^-_\theta})$,
but the resulting bounds can be extremely loose. In this section, we show a
different way to compute upper bounds to these quantities. Our experimental
results show that the approach we present here is practical and gives much
tighter bounds.

\begin{lemma}\label{lem:negbordvcdimupbound}
  Let $Q$ be a set of transactions and let $v$ be the maximum integer for which 
  there are at least $v$ transactions $\tau_1,\dotsc,\tau_v$ such that:
  \begin{enumerate}
    \item For any pair of transactions $(\tau_i,\tau_j)$, $i\neq j$, we have
      $\tau_i\not\subseteq\tau_j$ and $\tau_j\not\subseteq\tau_i$, and
    \item $\tau_v$ contains at least $2^{v-1}$ itemsets from
      $\mathcal{B}^-_\theta$.
  \end{enumerate}
  Then $\EVC(R_{\mathcal{B}^-_\theta},Q)\le v$.
\end{lemma}

\begin{proof}
  The first requirement guarantees that the set of transactions considered in
  the computation of $v$ could indeed theoretically be shattered. Assume that a
  subset $F$ of $Q$ contains two transactions $\tau'$ and $\tau''$ such that
  w.l.o.g.~$\tau'\subseteq\tau''$. Any itemset from $\mathcal{B}^-_\theta$
  appearing in $\tau'$ would also appear in $\tau''$, so there would not be any
  itemset $A\in\mathcal{B}^-_\theta$ such that $\tau''\in T(A)\cap F$ but
  $\tau'\not\in T(A)\cap F$, which would imply that $F$ can not be shattered.
  Hence sets that do not respect requirement 1.~should not be considered. This
  has the net effect of potentially result in a lower $v$, i.e., in a stricter
  bound to $\EVC(R_{\mathcal{B}^-_\theta},Q)\le v$.

  Let $\ell>v$ and consider a set $L$ of $\ell$ transactions from $Q$ such that
  requirement 1.~holds. Assume that $L$ is shattered by
  $R_{\mathcal{B}^-_\theta}$. Let $\tau$ be a transaction in $L$.
  The transactions $\tau$ belongs to $2^{\ell-1}$ subsets of $L$. Let
  $K\subseteq L$ be one of these subsets. Since $L$ is shattered, there exists an
  itemset $A\in\mathcal{B}^-_\theta$ such that $T(A)\cap L=K$. From this and the fact
  that $t\in K$, we have that $\tau\in T(A)$ or equivalently that
  $A\subseteq\tau$. Given that all the subsets $K$ containing $\tau$ are
  different, then also all the $T(A)$ such that $T(A)\cap L=K$ should be
  different, which in turn implies that all the itemsets
  $A\in\mathcal{B}^-_\theta$ should be different and that they should all
  appear in $\tau$. There are $2^{\ell-1}$ subsets $K$ containing $\tau$, then
  $\tau$ must contain at least $2^{\ell-1}$ itemsets from $\mathcal{B}^-_\theta$,
  and this holds for all $\ell$ transactions in $Q$. This is a contradiction
  because $\ell>v$ and $v$ is the maximum integer for which there are at least
  $v$ transactions containing at least $2^{v-1}$ itemsets from
  $\mathcal{B}^-_\theta$. Hence $L$ cannot be shattered and the thesis follows.
\end{proof}

An exact computation of $v$ defined as in Lemma~\ref{lem:negbordvcdimupbound} could
be extremely expensive as it would require to scan the transactions one by one
and compute the number of itemsets from $\mathcal{B}^-_\theta$ appearing in each
transaction. Moreover, the negative border $\mathcal{B}^-_\theta$ is \emph{not
known} in advance, so such a computation would not be possible. 

Let still assume we know $\mathcal{B}^-_\theta$. We now present an algorithm to
compute an upper bound to $v$. The algorithm will only need minor modifications
when the assumption to know $\mathcal{B}^-_\theta$ will be drop.

\paragraph{The Set-Union Knapsack Problem}
We first introduce the \emph{Set-Union Knapsack Problem} (SUKP). 
\begin{definition}[\citep{GoldschmidtNY94}]\label{def:sukp}
  Let $U=\{a_1,\dotsc,a_\ell\}$ be a set of elements and let
  $S=\{A_1,\dotsc,A_k\}$ be a set of subsets of $U$, i.e. $A_i\subseteq U$ for
  $1\le i\le k$. Each subset $A_i$, $1\le i\le k$, has an associated
  non-negative profit $\rho(A_i)\in\mathbb{R}^+$, and each element $a_j$, $1\le
  j\le\ell$ as an associate non-negative weight $w(a_j)\in\mathbb{R}^+$.
  Given a subset $S'\subseteq S$, we define the profit of $S'$ as
  $P(S')=\sum_{A_i\in S'}\rho(A_i)$. Let $U_{S'}=\cup_{A_i\in S'} A_i$. We
  define the weight of $S'$ as $W(S')=\sum_{a_j\in U_{S'}} w(a_j)$. 

  Given a non-negative parameter $c$ that we called \emph{capacity}, the
  \emph{Set-Union Knapsack Problem} (SUKP) requires to find the
  set $S^*\subseteq S$ which \emph{maximizes} $P(S')$ over all sets $S'$ for
  which $W(S')\le c$.
\end{definition}
The SUKP is NP-hard in the general case, but there are known restrictions for
which it can be solved in polynomial time using dynamic
programming~\citep{GoldschmidtNY94}. It is not clear whether our case falls into
one of these restrictions but we found that available optimization problem
solvers can compute the optimal solution reasonably fast even for very large
instances with thousands of elements and tens of thousands of subsets.

%{\bf XXX:} I wonder whether we may want to give some info on the problem, like
%NP-hardness, approximation algorithms, and stuff like that.

In our case, $U=\Itm$, $S=\mathcal{B}^-_\theta$, $w(a_j)=1 \forall a_j\in\Itm$,
and $p(A)=1\forall A\in\mathcal{B}^-_\theta$. The capacity $c$ will be a number
denoting a transaction length. It should be clear that the profit $P(S^*)$ of
the optimal solution $S^*$ to this problem will denote the maximum number of
itemsets from $\mathcal{B}^-_\theta$ that can appear in a transaction of length
$c$. Then, the SUKP can be formulated as a Binary Integer Program as follows.
\begin{description}
  \item[Constant:] Non-negative integer capacity $c$.
  \item[Variables:] \hfill
    \begin{itemize}
      \item Integer indicator variable $X_A$ for each
	$A\in\mathcal{B^-_\theta}$, taking values in $\{0,1\}$.
      \item Integer indicator variable $Y_a$ for each $a\in\Itm$, taking values
	in $\{0,1\}$.
    \end{itemize}
  \item[Objective:] maximize the function $\sum_{A\in\mathcal{B^-_\theta}}X_A$.
  \item[Constraints:] \hfill
    \begin{itemize}
      \item $\sum_{a\in\Itm} Y_a\le c$.
      \item $Y_a-X_A\ge 0$ for each itemset $A\in\mathcal{B^-_\theta}$ and for each item $a\in A$.
    \end{itemize}
\end{description}

\paragraph{Computing the upper bound}
Given a dataset $\Ds$ let now $\ell_1,\cdots,\ell_w$ be sequence of the
\emph{transaction lengths} of transactions in $\Ds$, i.e., for each value $\ell$
for which there is at least a transaction in $\Ds$ of length $\ell$, there is
one (and only one) index $i$, $1\le i\le w$ such that $\ell_i=\ell$. Assume that
the $\ell_i$'s are labelled in sorted decreasing order:
$\ell_1>\ell_2>\dotsb>\ell_w$. Let now $L_i$, $1\le i\le w$ be the maximum
number of transactions in $\Ds$ that have length at least $\ell_i$ and such that
for no two $\tau'$, $\tau''$ of them we have either $\tau'\subseteq\tau''$ or
$\tau''\subseteq\tau'$. The sequences $(\ell_i)_1^w$ and a sequence $(L_i^*)^w$
of upper bounds to $(L_i)_1^w$ can be computed efficiently with a scan of the
dataset .

%{\bf XXX:} shall we give more details on the algorithm to compute these
%sequences? It's booooring\ldots

Algorithm~\ref{alg:bindex} presents the pseudocode of
our method to compute an upper bound to $\EVC(R_{\mathcal{B}^-_\theta},\Ds)$.
The \texttt{solve\_SUKP($U,S,c$)} routine computes the optimal objective value for
the SUKP problem defined on the set of elements $U$, the set of subsets $S$,
with capacity $c$ and unitary profits and weights. 
\begin{algorithm}[ht]
  \SetKwInOut{Input}{Input}
  \SetKwInOut{Output}{Output}
  \SetKwFunction{SolveSUKP}{solve\_SUKP}
   \DontPrintSemicolon
  %\dontprintsemicolon
  \Input{a dataset $\Ds$ with transactions built on a ground set $\Itm$, a
  minimum probability mass threshold $\theta$.}
  \Output{the empirical b-index of $\Ds$ with respect to $\theta$}
  \For{$i\leftarrow 1$ to $w$} {
  $q\leftarrow$ \SolveSUKP{$\Itm,B^-_\theta,\ell_i$}\;
  $b\leftarrow\lfloor\log_2q\rfloor+1$\;
  \If{$L_i \ge b$} {
  \Return{b}\;
  }
  }
  \caption{computes an upper bound to $\EVC(R_{\mathcal{B}^-_\theta},\Ds)$}\label{alg:bindex}
\end{algorithm}

We can formulate the following lemma, whose correctness easily derives from
the discussion above.
\begin{lemma}\label{lem:assumptempvcdim}
  Let $b$ be the value returned by Algorithm~\ref{alg:bindex}. Then
  $\EVC(R_{\mathcal{B}^-_\theta},\Ds)\le b$.
\end{lemma}

Consider now the SUKP defined on $\Itm$ and
$\mathcal{B}^-_\theta$ with unitary weights and profits and with capacity
$c=|\Itm|-1$. Let $q^*$ be its optimal solution and let
$b^*=\lfloor\log_2q^*\rfloor+1$. The following lemma is an easy consequence of
this definition and of
Lemma~\ref{lem:assumptempvcdim}.

\begin{lemma}\label{lem:assmptvcdim}
  $\VC(R_{\mathcal{B}^-_\theta})\le b^*$.
\end{lemma}

\paragraph{Removing the assumption on $\mathcal{B}^-_\theta$}
Until now we assumed to know $\mathcal{B}^-_\theta$. This assumption is not
realizable in practice: if we knew the negative border exactly, we would also
know the collection of RFI's, as one uniquely identifies the other and
vice versa. We now remove this assumption and present a slight modification of
the method presented in the previous section to compute an upper bound to the
(empirical) VC-dimension of $R_{\mathcal{B}^-_\theta}$.

Let $B$ be a superset of $\mathcal{B}^-_\theta$. It should be evident that if we
replace $\mathcal{B}^-_\theta$ with $B$ in the SUKP problems described above,
then the optimal objective function values these modified problems will not be
smaller than the optimal objective function value of the original problems,
given that the original optimal solutions are feasible solutions of the modified
problems. 

Assume that we can compute a superset $B$ of $\mathcal{B}^-_\theta$ (we will
show how to do it in Section~\ref{sec:algo}). One drawback with the approach we
just outlined would be that the resulting bound to the (empirical) VC-dimension
could be much larger than the original bound. This is due to the fact that we
are not enforcing an additional constraint on the desired structure of the
optimal solution that is implicit when only considering $\mathcal{B}^-_\theta$:
the fact that the negative border is an \emph{antichain}.
Given a universe $\mathcal{U}$, an \emph{antichain} is a set
$\mathcal{F}$ of subsets of $\mathcal{U}$ such that for no two of these subsets
$X',X''\in\mathcal{F}$, we have $X'\subset X''$ or $X''\subset X'$.

We modify the formulation of the SUKP to restrict the set of feasible solutions
to only contain antichains. As in the original formulation fo the SUKP, let $U$
and $S$ be a set of elements, and a set of subsets of $U$, respectively. A \emph{chain}
on the members of $S$ is a subset $\{A_1,A_2,\dotsc,A_w\}\subseteq S$ such that
$A_1\subset A_2\subset\dotsb\subset A_w$. A \emph{maximal chain} on the members
of $S$ is a chain to which it is impossible to add any other member of $S$
without removing the chain property. Let $\mathcal{C}_S$ denote the set of all
the maximal chains on the members of $S$.
We modify the definition of the SUKP defined on $U$, $S$ by including
the following set of constraints:
\begin{itemize}
  \item $\forall C\in\mathcal{C}_S, \sum_{A\in C} X_A\le 1$.
\end{itemize}
We call this modified version of the SUKP the \emph{Antichain SUKP} (ASUKP).

Let $B$ be a superset of $\mathcal{B}^-_\theta$. We modify
Algorithm~\ref{alg:bindex} by replacing the call to \texttt{solve\_SUKP($\Itm,B^-_\theta,\ell_i$)}
with a call to the routine \texttt{solve\_ASUKP($\Itm,B,\ell_i$}, which computes the
optimal value of the objective function for the ASUKP defined on $\Itm$ and $B$, with
capacity $\ell_i$ and unitary weights and profits. It should be evident that
this value is not smaller than the value one would obtain using the original
call to \texttt{solve\_SUKP($\Itm,B^-_\theta,\ell_i$)} because the optimal
solution to the problem corresponding to this call is still a feasible solution
for the problem corresponding to \texttt{solve\_ASUKP($\Itm,B,\ell_i$)}.

\begin{definition}\label{def:empbindex}
  Let $b$ be the solution returned by the modified Algorithm~\ref{alg:bindex}.
  We call $b$ the \emph{empirical b-index of $\mathcal{B}^-_\theta$ on $\Ds$
  using $B$} and denote it with $\mathsf{eb}(\mathcal{B}^-_\theta,\Ds,B)$.
\end{definition}

This lemma then follows easily from the above discussion and
Lemma~\ref{lem:assumptempvcdim}.
\begin{lemma}\label{lem:empbindexbound}
  $\EVC(R_{\mathcal{B}^-_\theta},\Ds)\le\mathsf{eb}(\mathcal{B}^-_\theta,\Ds,B)$.
\end{lemma}

We now proceed to bound $\VC(R_{\mathcal{B}^-_\theta})$ in a way similar to what
we did when we assumed to know $\mathcal{B}^-_\theta$.
Let $q^*$ be the optimal value of the objective function for the ASUKP defined on
$\Itm$ and $B$ with capacity $|\Itm|-1$ and unitary weights and profits. Let
$b^*=\lfloor\log_2 q^*\rfloor+1$.

\begin{definition}\label{def:bindex}
  We call $b^*$ the \emph{b-index of $\mathcal{B}^-_\theta$ using $B$} and
  denote it with $\mathsf{b}(\mathcal{B}^-_\theta,B)$.
\end{definition}

We can now conclude with the following lemma, which is a direct consequence of
the above discussion and Lemma~\ref{lem:assmptvcdim}.
\begin{lemma}\label{lem:bindexbound}
  $\VC(R_{\mathcal{B}^-_\theta})\le\mathsf{b}(\mathcal{B}^-_\theta,B)$.
\end{lemma}

\iffalse
\subsubsection{A lower bound to the VC-dimension of $R_{\mathcal{B}^-_\theta}$}
One may ask how tight the (empirical) b-index is as an upper bound to the
(empirical) VC-dimension of the range set of the negative border of RFI's. We
have the following results, showing that the bound is optimal for a family\dots
({\bf XXX:} A family of what? Datasets? Distributions (distribution on what?)
\begin{lemma}
  There are datasets and distributions for which the bound is strict
\end{lemma}
\begin{proof}
  TBD
\end{proof}
\fi

\subsection{The statistical test}\label{sec:algo}
In this section we present our method to compute a collection of RFI's while
controlling its FWER. We then prove its correctness using the concepts and
results defined in the previous section.

The intuition behind our statistical test is extremely simple: we first use the
results from Section~\ref{sec:rangeitemsets} to compute a lowered frequency threshold
$\theta-\varepsilon'$ such that the collection of FI's at that threshold is a
superset of the collection of RFI's. We then use this collection as the basis to
compute a superset to the negative border of the collection of RFI's. From here,
using the results from Section~\ref{sec:bindexub}, we proceed to compute upper
bounds to the (empirical) VC-dimension of the range set of the negative border.
Finally, we compute another offset parameter $\varepsilon''$ so that the FI's
with frequency in the dataset at least $\theta+\varepsilon''$ are all RFI's,
with probability at least $1-\delta$. The pseudocode for the statistical test is
presented in Algorithm~\ref{alg:pseudocode}.

\begin{algorithm}[ht]
  \SetKwInOut{Input}{Input}
  \SetKwInOut{Output}{Output}
  \SetKwFunction{GetEpsilonVC}{get\_epsilon\_VC}
  \SetKwFunction{GetEpsilonEmpVC}{get\_epsilon\_empVC}
  \SetKwFunction{GetDIndex}{get\_d-index}
  \SetKwFunction{GetEmpVCDimUpperBound}{get\_emp\_b-index}
  \SetKwFunction{GetVCDimUpperBound}{get\_b-index}
  \DontPrintSemicolon
  %\dontprintsemicolon
  \Input{a dataset $\Ds$, a minimum probability mass threshold $\theta$, a FWER
    bound $\delta$.}
  \Output{a collection of itemsets $A$ with $r(A)\ge\theta$ such that FWER $<\delta$.}
  $\delta',\delta''\leftarrow 1-\sqrt{1-\delta}$\; 
  $d_1 = |\Itm|-1$\;
  $d_2 = \GetDIndex{$\Ds$}$\;
  $\varepsilon'_1\leftarrow$ \GetEpsilonVC{$d_1, |\Ds|, \delta'$}\;\label{algline:epsVC}
  $\varepsilon'_2\leftarrow$ \GetEpsilonEmpVC{$d_2, |\Ds|, \delta'$}\;\label{algline:epsEVC}
  $\varepsilon'\leftarrow\min\{\varepsilon'_1,\varepsilon'_2\}$\;
  $\mathcal{X}\leftarrow\FI(\Ds,\Itm,\theta-\varepsilon')$\;\label{algline:mining}
  $\mathcal{Z}\leftarrow (\mathcal{X} \setminus \FI(\Ds,\Itm,\theta+\varepsilon')) \cup
  \mathcal{B}^-(\mathcal{X})$.\;\label{algline:candidates}
  $b_1\leftarrow$
  \GetVCDimUpperBound{$|\Itm|-1,\mathcal{Z}$}\;\label{algline:ASUKP}
  $b_2\leftarrow$
  \GetEmpVCDimUpperBound{$\Itm,\Ds,\mathcal{Z}$}\;\label{algline:empASUKP}
  $\varepsilon''_1\leftarrow$ \GetEpsilonVC{$b_1, |\Ds|, \delta''$}\;
  $\varepsilon''_2 \leftarrow$ \GetEpsilonEmpVC{$b_2, |\Ds|, \delta''$}\;
  $\varepsilon''\leftarrow\min\{\varepsilon''_1,\varepsilon''_2\}$\;
  \Return{$\FI\left(\Ds,\Itm,\theta+\varepsilon''\right)$}\;
  \caption{A statistical test for RFI's}\label{alg:pseudocode}
\end{algorithm}

\sloppy
The routine \texttt{get\_epsilon\_VC} in line~\ref{algline:epsVC} (resp.~\texttt{get\_epsilon\_empVC} in line~line~\ref{algline:epsEVC}) takes
as input an upper bound $d^*$ to the VC-dimension of $R_\Itm$ (resp.~to the empirical
VC-dimension of $R_\Itm$ on $\Ds$), a dataset size $|\Ds|^*$, and a real
$\delta^*\in(0,1)$, and uses~\eqref{eq:vceapprox} (resp.~\eqref{eq:evceapprox})
to compute a value $\varepsilon^*$ such that a dataset of size  $|\Ds|$ is a
$\varepsilon^*$-approximation to a range set with VC-dimension (resp.~with
empirical VC-dimension) at most $d^*$ with probability at least $1-\delta^*$.

The routine \texttt{get\_d-index} computes (an upper bound to) the d-index of a dataset, as
described in~\citep{RiondatoU12}. The routine
\texttt{get\_b-index} (resp.~\texttt{get\_emp\_b-index}) computes
$\mathsf{b}(\mathcal{B}^-(\RFI(p,\Itm,\theta)),\mathcal{Z}^*)$
(resp.~$\mathsf{eb}(\mathcal{B}^-(\RFI(p,\Itm,\theta)),\Ds,\mathcal{Z}^*)$),  the b-index
of $\mathcal{B}^-(\RFI(p,\Itm,\theta))$ (resp.~the empirical b-index of
$\mathcal{B}^-(\RFI(p,\Itm,\theta))$ on $\Ds$) using a superset $\mathcal{Z}^*$
of $\mathcal{B}^-(\RFI(p,\Itm,\theta))$. We described these computations in
Section~\ref{sec:bindexub}.

\begin{lemma}\label{lem:firsteapprox}
  With probability at least $1-\delta'$, $\Ds$ is an
  $\varepsilon'$-approximation to $R_\Itm$.
\end{lemma}

\begin{proof}
  Suppose $\varepsilon'=\varepsilon'_1$. From Theorem~\ref{thm:vcdimubfirst} we
  know that $\VC(R_\Itm)\le d_1 (= |\Itm|-1)$.
  The thesis follows from the definition of $\varepsilon'_1$ (through the
  routine \texttt{get\_epsilon\_VC}$(\cdot)$) and Theorem~\ref{thm:eapprox}.

  Suppose instead that $\varepsilon'=\varepsilon'_2$. From
  Theorem~\ref{thm:empvcdimubfirst} we know that $\EVC(R_\Itm,\Ds)\le d_2$,
  given that $d_2$ is (an upper bound to) the d-index of $\Ds$. The thesis follows from the
  definition of $\varepsilon'_2$ (through the routine
  \texttt{get\_epsilon\_empVC}$(\cdot)$) and Theorem~\ref{thm:eapproxempir}.
\end{proof}

\begin{lemma}\label{lem:negbordersuperset}
  If $\Ds$ is a $\varepsilon'$-approximation to $R_\Itm$, then
  $\mathcal{B}^-(\RFI(p,\Itm,\theta))\subseteq\mathcal{Z}$, where $\mathcal{Z}$
  is defined as in line~\ref{algline:candidates} of
  Algorithm~\ref{alg:pseudocode}.
\end{lemma}

\begin{proof}
From the hypothesis we have that, for every itemset $A\in\mathcal{Z}$,
$f_{\Ds}(A) \in [r(X) - \varepsilon'; r(X)+\varepsilon']$. Consider
$B\in\mathcal{B}^-(\RFI(p,\Itm,\theta))$. This by definition implies
$r(B)<\theta$. 

If $f_\Ds(B) \ge \theta - \varepsilon'$, since $r(B) < \theta$,
then $f_\Ds(B) < \theta + \varepsilon'$, that is $B \in\mathcal{Z}$.

Assume instead that $f_\Ds(B) < \theta - \varepsilon'$. Since $B \in
\mathcal{B}^-(\RFI(p,\Itm,\theta))$, then for all $C \in
2^B\setminus\{B\}$ we have $r(C) \ge \theta$, therefore for all $C \in
2^B\setminus\{B\}$ we have $f_\Ds(C) \ge \theta - \varepsilon'$, which implies $B \in
\mathcal{B}^-(\FI(\Ds\,\Itm,\theta-\varepsilon'))$. Since
$\mathcal{B}^-(\FI(\Ds\,\Itm,\theta-\varepsilon'))\subset\mathcal{Z}$, then
$B\in\mathcal{Z}$.
\end{proof}

\begin{lemma}\label{lem:upboundvc}
  Assume that $R_{\mathcal{B}^-(\RFI(p,\Itm,\theta))}\subseteq\mathcal{Z}$; then
  $\VC(R_{\mathcal{B}^-(\RFI(p,\Itm,\theta)}))\le b_1$ and
  $\EVC(R_{\mathcal{B}^-(\RFI(p,\Itm,\theta)}),\Ds)\le b_2$.
\end{lemma}

\begin{proof}
  The result follows from Lemma~\ref{lem:bindexbound} and
  Lemma~\ref{lem:empbindexbound}.
\end{proof}

\begin{lemma}\label{len:secondeapprox}
  With probability at least $1-\delta$, $\Ds$ is an
  $\varepsilon''$-approximation to $R_{\mathcal{B}^-(\RFI(p,\Itm,\theta))}$.
\end{lemma}

\begin{proof}
  Assume that $\Ds$ is an $\varepsilon_1$-approximation to $R_\Itm$.
  From Lemma~\ref{lem:firsteapprox} we know that this happens with probability
  at least $1-\delta'$. Then, Lemma~\ref{lem:negbordersuperset} holds, and
  therefore Lemma~\ref{lem:upboundvc} also holds. Assume now that
  $\varepsilon''=\varepsilon''_1$. Then the thesis follows from the
  definitions of $\varepsilon''_1$ and of $\delta''$, and from
  Theorem~\ref{thm:eapprox}. If instead $\varepsilon''=\varepsilon''_2$, the thesis follows
  from the definitions of $\varepsilon''_2$ and of $\delta''$ and from
  Theorem~\ref{thm:eapproxempir}.
\end{proof}
  
We conclude the analysis of the correctness of our method with a corollary which
follows from the property of an $\varepsilon''$-approximation
(Def.~\ref{def:eapprox}).
\begin{corollary}
  The FWER of the statistical test implemented by Algorithm~\ref{alg:pseudocode} is at most $\delta$.
\end{corollary}


