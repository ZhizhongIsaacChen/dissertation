%\subsection{Method 2: a full-dataset approach}\label{sec:fulltest}
%We now present a method for identifying the threshold $\hat{\theta}$ to be used 
%to mine TFI's.
The intuition behind the method is the following. 
Let $\mathcal{B}$ be
the \emph{negative border} of $\TFI(\prob,\Itm,\theta)$, that is the set of itemsets
not in $\TFI(\prob,\Itm,\theta)$ but such that all their proper subsets are in
$\TFI(\prob,\Itm,\theta)$. If we can find an $\varepsilon$ such that $\Ds$ is an
$\varepsilon$-approximation to $(2^\Itm,\range(\mathcal{B}),\pi)$ then %with probability at least $1-\delta$, then, if this is the case, 
we have that any itemset
$A\in\mathcal{B}$ has a frequency $f_\Ds(A)$ in $\Ds$ less than
$\hat{\theta}=\theta+\varepsilon$, given that it must be $\tfreq(A)<\theta$. By the
antimonotonicity property of the frequency, the same holds for all itemsets that
are supersets of those in $\mathcal{B}$. Hence, the only itemsets that can have
frequency in $\Ds$ greater or equal to $\hat{\theta}=\theta+\varepsilon$ are
those with true frequency at least $\theta$. In the following paragraphs we show
how to compute $\varepsilon$.

Let $\delta_1$ and $\delta_2$ be such that $(1-\delta_1)(1-\delta_2)\ge
1-\delta$. Let $(2^\Itm,\range(2^\Itm)$ be the range space of all itemsets.
We use Corol.~\ref{coroll:vcdimubfirst} (resp.~Thm.~\ref{coroll:empvcdimubfirst}) to
compute an upper bound $d'$ to $\VC\left((2^\Itm,\range(2^\Itm))\right)$ (resp.~$d''$ to
$\EVC(\Ds,\range(2^\Itm)$). Then we can use $d'$ in Thm.~\ref{thm:eapprox} (resp.~$d''$ in
Thm~\ref{thm:eapproxempir}) to compute an $\varepsilon_1'$ (resp.~an
$\varepsilon_1''$) such that $\Ds$ is, with probability at
least $1-\delta_1$, an $\varepsilon_1'$-approximation
(resp.~$\varepsilon_1''$-approximation) to $(2^\Itm,\range(2^\Itm),\prob)$.
\begin{fact}
%Then, if we 
Let
$\varepsilon_1=\min\{\varepsilon_1',\varepsilon_1''\}$. %, 
%$\Ds$ is, with probability at least $1-\delta_1$, an
%$\varepsilon_1$-approximation to $(\range(2^\Itm),\prob)$.
With probability at least $1-\delta_1$, $\Ds$ is an
$\varepsilon_1$-approximation to $(2^\Itm,\range(2^\Itm),\prob)$.
\end{fact}

%Following the same steps as in the first phase of Method 1, we can
%find an $\varepsilon_1$ such that $\Ds$ is an $\varepsilon_1$-approximation for
%$(\range(2^\Itm),\prob)$ with probability at
%least $1-\delta_1$.
We want to find an upper bound the (empirical) VC-dimension of
$(2^\Itm,\range(\mathcal{B})$. To this end, we use the fact that the negative border of a
collection of itemsets is a \emph{maximal
antichain} on $2^\Itm$. %, that is, a collection of sets from $2^\Itm$ such that
%for no two of them, one of them is included in the other and such that is not a
%proper subset of any other antichain.
Let now $\mathcal{W}$ be the \emph{negative
border} of $\mathcal{C}_1=\FI(\Ds,\Itm,\theta-\varepsilon_1)$, 
$\mathcal{G}=\{A\subseteq\Itm ~:~ \theta-\varepsilon_1\le
f_\Ds(A)<\theta+\varepsilon_1\}$, and $\mathcal{F}=\mathcal{G}\cup\mathcal{W}$.

\begin{lemma}\label{lem:antichains}
  Let $\mathcal{Y}$ be the set of maximal antichains in $\mathcal{F}$. If
  $\Ds$ is an $\varepsilon_1$-approximation to $(2^\Itm,\range(2^\Itm),\prob)$, then
  \begin{enumerate}
    \item
      $\max_{\mathcal{A}\in\mathcal{Y}}\EVC(\Ds,\range(\mathcal{A}))\ge\EVC(\Ds,\range(\mathcal{B}))$,
      and
    \item
      $\max_{\mathcal{A}\in\mathcal{Y}}\VC\left((2^\Itm,\range(\mathcal{A}))\right)\ge\VC\left((2^\Itm,\range(\mathcal{B}))\right)$.
  \end{enumerate}
\end{lemma}
\ifarxiv
\begin{proof}
  Given %Assume 
  that $\Ds$ is an $\varepsilon_1$-approximation to $(\range(2^\Itm),\prob)$, %From the definition of $\varepsilon_1$ this happens
  %with probability at least $1-\delta$. 
  then %Then
  $\TFI(\prob,\Itm,\theta)\subseteq\mathcal{G}$. %\cup\mathcal{C}_1$. 
  From this and
  the definition of negative border and of $\mathcal{F}$, we have that
  $\mathcal{B}\subseteq\mathcal{F}$. Since $\mathcal{B}$ is a maximal
  antichain, then $\mathcal{B}\in\mathcal{Y}$. Hence the thesis.
  \qed
\end{proof}
\fi

In order to compute upper bounds to
$\VC\left((2^\Itm,\range(\mathcal{B}))\right)$ and
$\EVC(\range(\mathcal{B}),\Ds)$ we can solve slightly modified SUKPs 
associated to $\mathcal{F}$ with the additional constraint that the
optimal solution, which is a collection of itemsets, \emph{must be a maximal
antichain}. Lemma~\ref{lem:sukpevc} still holds even for the solutions of these
modified SUKPs. Using these bounds in Thms.~\ref{thm:eapprox}
and~\ref{thm:eapproxempir}, we compute an $\varepsilon_2$ such that, with
probability at least $1-\delta_2$, $\Ds$ is an $\varepsilon_2$-approximation to
$(\range(\mathcal{B}),\prob)$. Let 
\[
\hat{\theta}=\theta+\varepsilon_2\enspace.\]

%The method returns the collection of itemsets
%$\FI(\Ds,\Itm,\theta+\varepsilon_2)$, and we prove the following. %(proof in the full version~\citep{RiondatoV13}).

\begin{theorem}\label{lem:vcfull}
With probability at least $1-\delta$, %$\FI(\Ds,\Itm,\hat{\theta})$ contains no
%itemset $A\not\in\TFI(\prob,\Itm,\theta)$:
$\FI(\Ds,\Itm,\hat\theta)$ contains no false positives:
\[
\Pr\left(\FI(\Ds,\Itm,\hat\theta)\subseteq\TFI(\prob,\Itm,\theta)\right)\ge 1-\delta\enspace.\]
%\[
%\Pr(\exists A \mbox{ s.t. } f_\Ds(A)\ge\hat\theta \mbox{ and }
%\tfreq(A)<\theta)\le\delta\enspace.\]
%   The probability that $\FI(\Ds,\Itm,\hat{\theta})$
%contains any itemset $A\not\in\TFI(\prob,\Itm,\theta)$ at most $\delta$:
%\[
%\Pr(\exists A \mbox{ s.t. } f_\Ds(A)\ge\hat\theta \mbox{ and }
%\tfreq(A)<\theta)\le\delta\enspace.\]
\end{theorem}
\ifarxiv
\begin{proof}
  Consider the two events $\mathsf{E}_1$=``$\Ds$ is an
  $\varepsilon_1$-approximation for $(\range(2^\Itm),\prob)$'' and
  $\mathsf{E}_2=$``$\Ds$ is an
  $\varepsilon_2$-approximation for $(\range(\mathcal{B}),\prob)$''. From
  the above discussion and the definition of $\delta_1$ and $\delta_2$ it
  follows that the event $\mathsf{E}=\mathsf{E}_1\cap\mathsf{E}_1$ occurs with
  probability at least $1-\delta$. Suppose from now on that indeed $\mathsf{E}$
  occurs.

  Since $\mathsf{E}_1$ occurs, then Lemma~\ref{lem:antichains}
  holds, and the bounds we compute by solving the modified SUKP problems are
  indeed bounds to $\VC\left((2^\Itm,\range(\mathcal{B}))\right)$ and
  $\EVC(\range(\mathcal{B},\Ds))$. %Then the computation of $\varepsilon_2$ is valid. 
  Since $\mathsf{E}_2$ also occurs, then for any $A\in\mathcal{B}$ we
  have $|\tfreq(A)-f_\Ds(A)|\le\varepsilon_2$, but given that $\tfreq(A)<\theta$
  because the elements of $\mathcal{B}$ are not TFIs, then we have
  $f_\Ds(A)<\theta+\varepsilon_2$. Because of the antimonotonicity property
  of the frequency and the definition of $\mathcal{B}$, this holds for any
  itemset that is not in $\TFI(\prob,\Itm,\theta)$. Hence, the only itemsets that can have a
  frequency in $\Ds$ at least $\hat{\theta}=\theta+\varepsilon_2$ are the TFIs, so
  $\FI(\Ds,\Itm,\hat{\theta})\subseteq\TFI(\prob,\Itm,\theta)$, which concludes
  our proof.
  \qed
\end{proof}

\paragraph{Exploiting additional knowledge about $\prob$.} 
Our algorithm is completely \emph{distribution-free}, i.e., it does not require
any assumption about the unknown distribution $\prob$. On the other hand, when
information about $\prob$ is available, our method can exploit it to achieve
better performances in terms of running time, practicality, and accuracy. 
For example, in most applications $\prob$ will not generate any transaction
longer than some upper bound $\ell\ll|\Itm|$, and this is know. 
Consider for example an online
marketplace like Amazon: it is extremely unlikely (if not humanly impossible)
that a single customer buys one of each available product. Indeed, given the
hundred of thousands of items on sale, it is safe to assume that all the
transactions will contains at most $\ell$ items, for $\ell\ll|\Itm|$. Other
times, like in an online survey, it is the nature of the process that limits the
number of items in a transaction, in this case the number of questions. A
different kind of information about the generative process may consists in
knowing that some combination of items may never occur, because ``forbidden'' in
some wide sense. Other examples are possible. All these pieces of information
can be used to compute better (i.e., stricter) upper bounds to the VC-dimension
$\VC\left((2^\Itm,\range(2^\Itm))\right)$. For example, if we know that $\prob$ will never generate
transactions with more than $\ell$ items, we can safely say that
$\VC\left((2^\Itm,\range(2^\Itm))\right)\le \ell$, a much stricter bound than $|\Itm|-1$ from
Corol.~\ref{coroll:vcdimubfirst}. This may result in a smaller $\varepsilon_1$, a smaller 
$\varepsilon$, and a smaller $\hat\theta$, which allows to produce
more TFIs in the output collection. In the experimental evaluation, we show the
positive impact of including additional information may on the performances of
our algorithm.
\fi
