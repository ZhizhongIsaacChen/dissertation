\documentclass[twoside,leqno,twocolumn]{article}  
%\usepackage[ruled,linesnumbered]{algorithm2e}
\usepackage{ltexpprt} 
\usepackage{balance}
\usepackage[numbers,sort&compress,comma]{natbib}
\usepackage[pdftex,bookmarks=true,pdfstartview={FitH},pdfauthor={Matteo Riondato
and Fabio Vandin},pdftitle={Finding the True Frequent Itemsets},pdfsubject={Frequent Itemsets
Mining}]{hyperref}
\usepackage[T1]{fontenc}
\usepackage{ae,aecompl}
\usepackage[utf8x]{inputenc}
\usepackage{amsfonts}
\usepackage{amsmath}
\usepackage{amssymb}
\usepackage{dsfont}
%\usepackage{amsthm}
\usepackage{mathtools}
\usepackage{graphicx}
%\usepackage{subfig}
\usepackage{multirow}
%\usepackage{bigstrut}
\usepackage{url}
\usepackage{mdwlist}
\usepackage{booktabs}
%\usepackage{savesym}
%\savesymbol{algorithm}
%\savesymbol{endalgorithm}
\usepackage{datetime}
\usepackage{times}
\usepackage{mathptm}

%\renewcommand\bibnumfmt[1]{[#1]} 
%\renewcommand{\bibsection}{\section{\refname}}

\newif\ifarxiv
\arxivfalse
%\arxivtrue

%\def\newblock{\hskip .11em plus .33em minus .07em}
%\DeclareCaptionType{copyrightbox}

\def\qed{}
\def\Sam{{\cal S}}
\def\Ds{{\cal D}}
\def\Itm{{\cal I}}
\def\FI{\mathsf{FI}}
\def\TFI{\mathsf{TFI}}
\def\VC{\mathsf{VC}}
\def\EVC{\mathsf{EVC}}
\def\AR{\mathsf{AR}}
\def\TOPK{\mathsf{TOPK}}
\def\RFI{\mathsf{RFI}}
\def\prob{\pi}
\def\tfreq{t_\prob}
\def\range{\mathcal{R}}
\def\XXX{{\bf XXX}}
\def\MR{{\bf MR}}
\def\FV{{\bf FV}}
\def\eb{\mathsf{eb}}
\def\b{\mathsf{b}}

%\newtheorem{corollary}{Corollary}
%\newtheorem{conj}{Conjecture}
%\newtheorem{lemma}{Lemma}
%\newtheorem{claim}{Claim}
%\newtheorem{theorem}{Theorem}

%\theoremstyle{definition}
\newtheorem{definition}{Definition}


\begin{document}
\title{Additional Material}
\section{Proofs of Theorems and Lemmas}

\textsc{Theorem 4.1.} Let $\mathcal{C}$ be a collection of itemsets and let $\Ds$ be a dataset. Let
  $d$ be the maximum integer for which there are at least $d$
  transactions $\tau_1,\dotsc,\tau_d\in \Ds$ such that the set
  $\{\tau_1,\dotsc,\tau_d\}$ is an antichain, and each $\tau_i$, $1\le i\le d$
  contains at least $2^{d-1}$ itemsets from $\mathcal{C}$. 
  %\begin{enumerate*}
  %  \item 
  %    The set $\{\tau_1,\dotsc,\tau_d\}$ is an antichain, and
  %    %For any pair of transactions $(\tau_i,\tau_j)$, $i\neq j$, we have
  %    %$\tau_i\not\subseteq\tau_j$ and $\tau_j\not\subseteq\tau_i$, and
  %  \item Each $\tau_i$, $1\le i\le d$ contains at least $2^{d-1}$ itemsets from
  %    $\mathcal{C}$.
  %\end{enumerate*}
  Then $\EVC(\range(\mathcal{C}),\Ds)\le d$.

\begin{proof}
  The antichain requirement guarantees that the set of transactions considered in
  the computation of $d$ could indeed theoretically be shattered. Assume that a
  subset $\mathcal{F}$ of $\Ds$ contains two transactions $\tau'$ and $\tau''$
  such that $\tau'\subseteq\tau''$. Any itemset from $\mathcal{C}$
  appearing in $\tau'$ would also appear in $\tau''$, so there would not be any
  itemset $A\in\mathcal{C}$ such that $\tau''\in T(A)\cap F$ but
  $\tau'\not\in T(A)\cap \mathcal{F}$, which would imply that $\mathcal{F}$ can
  not be shattered. Hence sets that are not antichains should not be
  considered. This has the net effect of potentially resulting in a lower $d$,
  i.e., in a stricter upper bound to $\EVC(\range(\mathcal{C}),\Ds)$.

  Let now $\ell>d$ and consider a set $\mathcal{L}$ of $\ell$ transactions from
  $\Ds$ that is an antichain. Assume that $\mathcal{L}$ is shattered by
  $\range(\mathcal{C})$. Let $\tau$ be a transaction in $\mathcal{L}$.
  The transactions $\tau$ belongs to $2^{\ell-1}$ subsets of $L$. Let
  $\mathcal{K}\subseteq \mathcal{L}$ be one of these subsets. Since
  $\mathcal{L}$ is shattered, there exists an itemset $A\in\mathcal{C}$ such
  that $T(A)\cap \mathcal{L}=\mathcal{K}$. From this and the fact
  that $t\in \mathcal{K}$, we have that $\tau\in T(A)$ or equivalently that
  $A\subseteq\tau$. Given that all the subsets $\mathcal{K}\subseteq\mathcal{L}$
  containing $\tau$ are different, then also all the $T(A)$'s such that
  $T(A)\cap \mathcal{L}=\mathcal{K}$ should be
  different, which in turn implies that all the itemsets
  $A$ should be different and that they should all appear in $\tau$. There are
  $2^{\ell-1}$ subsets $\mathcal{K}$ of $\mathcal{L}$ containing $\tau$,
  therefore $\tau$ must contain at least $2^{\ell-1}$ itemsets from
  $\mathcal{C}$, and this holds for all $\ell$ transactions in $\mathcal{L}$. This is a
  contradiction because $\ell>d$ and $d$ is the
  maximum integer for which there are at least $d$ transactions containing at
  least $2^{d-1}$ itemsets from $\mathcal{C}$. Hence $\mathcal{L}$ cannot be shattered and
  the thesis follows.
\end{proof}

\textsc{Lemma 4.1.} Let $j$ be the minimum integer for which $b_i\le L_i$. Then
  $\EVC(\mathcal{C},\Ds)\le b_j$. %We call $b_j$ the \emph{empirical b-index of
  %$\mathcal{C}$ on $\Ds$} and denote it as $\eb(\mathcal{C},\Ds)$

\begin{proof}
  If $b_j\le L_j$, then there are at least $b_j$ transactions which can contain
  $2^{b_j-1}$ itemsets from $\mathcal{C}$ and this is the maximum $b_i$ for
  which it happens, because the sequence $b_1,b_2,\dotsc,b_w$ is sorted in
  decreasing order, given that the sequence $q_1,q_2,\dotsc,q_w$ is. Then $b_j$
  satisfies the conditions of Thm.~4.1. Hence
  $\EVC(\mathcal{C},\Ds)\le b_j$.
\end{proof}

\textsc{Lemma 5.1.} Let $\mathcal{Y}$ be the set of maximal antichains in $\mathcal{F}$. If
  $\Ds$ is an $\varepsilon_1$-approximation to $(\range(2^\Itm),\prob)$, then
  \begin{enumerate}
    \item
      $\max_{\mathcal{A}\in\mathcal{Y}}\EVC(\range(\mathcal{A}),\Ds)\ge\EVC(\range(\mathcal{B}),\Ds)$,
      and
    \item
      $\max_{\mathcal{A}\in\mathcal{Y}}\VC(\range(\mathcal{A}))\ge\VC(\range(\mathcal{B}))$.
  \end{enumerate}
\begin{proof}
  Given %Assume 
  that $\Ds$ is an $\varepsilon_1$-approximation to $(\range(2^\Itm),\prob)$, %From the definition of $\varepsilon_1$ this happens
  %with probability at least $1-\delta$. 
  then %Then
  $\TFI(\prob,\Itm,\theta)\subseteq\mathcal{G}\cup\mathcal{C}_1$. From this and
  the definition of negative border and of $\mathcal{F}$, we have that
  $\mathcal{B})\subseteq\mathcal{F}$. Since $\mathcal{B}$ is a maximal
  antichain, then $\mathcal{B}\in\mathcal{Y}$. Hence the thesis.
\end{proof}

\textsc{Theorem 5.1.}
With probability at least $1-\delta$, %$\FI(\Ds,\Itm,\hat{\theta})$ contains no
%itemset $A\not\in\TFI(\prob,\Itm,\theta)$:
$\FI(\Ds,\Itm,\hat\theta)$ contains no false positives:
\[
\Pr\left(\FI(\Ds,\Itm,\hat\theta)\subseteq\TFI(\prob,\Itm,\theta)\right)\ge 1-\delta\enspace.\]
\begin{proof}
  Consider the two events $\mathsf{E}_1$=``$\Ds$ is an
  $\varepsilon_1$-approximation for $(\range(2^\Itm),\prob)$'' and
  $\mathsf{E}_2=$``$\Ds$ is an
  $\varepsilon_2$-approximation for $(\range(\mathcal{B}),\prob)$''. From
  the above discussion and the definition of $\delta_1$ and $\delta_2$ it
  follows that the event $\mathsf{E}=\mathsf{E}_1\cap\mathsf{E}_1$ occurs with
  probability at least $1-\delta$. Suppose from now on that indeed $\mathsf{E}$
  occurs.

  Since $\mathsf{E}_1$ occurs, then Lemma 5.1
  holds, and the bounds we compute by solving the modified SUKP problems are
  indeed bounds to $\VC(\range(\mathcal{B}))$ and
  $\EVC(\range(\mathcal{B},\Ds))$. %Then the computation of $\varepsilon_2$ is valid. 
  Since $\mathsf{E}_2$ also occurs, then for any $A\in\mathcal{B}$ we
  have $|\tfreq(A)-f_\Ds(A)|\le\varepsilon_2$, but given that $\tfreq(A)<\theta$
  because the elements of $\mathcal{B}$ are not TFIs, then we have
  $f_\Ds(A)<\theta+\varepsilon_2$. Because of the antimonotonicity property
  of the frequency and the definition of $\mathcal{B}$, this holds for any
  itemset that is not in $\TFI(\prob,\Itm,\theta)$. Hence, the only itemsets that can have a
  frequency in $\Ds$ at least $\hat{\theta}=\theta+\varepsilon_2$ are the TFIs, so
  $\FI(\Ds,\Itm,\hat{\theta})\subseteq\TFI(\prob,\Itm,\theta)$, which concludes
  our proof.
\end{proof}


\end{document}

