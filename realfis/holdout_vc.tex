\documentclass{article}
\usepackage{amsthm}

\newcommand{\Sam}{{\cal S}}
\newcommand{\Ds}{{\cal D}}
\def\Itm{{\cal I}}
\def\TOPK{\mathsf{TOPK}}
\def\FI{\mathsf{FI}}
\def\RFI{\mathsf{RFI}}
\def\AR{\mathsf{AR}}
\def\VC{\mathsf{VC}}
\def\EVC{\mathsf{EVC}}

\newtheorem{lemma}{Lemma}

\begin{document}
\title{ }
\author{ }

This method is inspired by the holdout statistical test presented by Webb. The
dataset is randomly split in two portions, not necessarily of the same size: an
\emph{exploratory} part $\Ds_\mathrm{e}$ and an
\emph{evaluation} part $\Ds_\mathrm{v}$. 
%The method does not require the sizes $|\Ds_\mathrm{e}|$ and $|\Ds_\mathrm{v}|$ to be the same.

The method works in two phases. The intuition behind it is the following: in the
first phase we use the exploratory part
$\Ds_\mathrm{e}$ to identify a small set of hypotheses (i.e.~candidate TFI's)
which will then, in the second phase,be tested on the evaluation part $\Ds_\mathrm{v}$. 

Let $\delta_1=\delta_2=1-\sqrt{1-\delta}$. Given that $\Ds_\mathrm{e}$ is still
a collection of i.i.d.~samples from the generative distribution $p$, then
through Thm.~\ref{thm:XXX} we know that there exists an $\varepsilon'$ such that
$\Ds_\mathrm{e}$ is a $\varepsilon'$-approximation to XXX the range space of all
itemsets with probability at least $1-\delta_1$. 

In the first phase of the set we compute the collection of itemsets
$\mathcal{T}_\mathrm{e}=\{X\subseteq\Itm ~:
f_{\Ds_\mathrm{e}}(X)\ge\theta+\varepsilon'\}$
and $\mathcal{G}=\{X\subseteq\Itm \theta\ge
f_{\Ds_\mathrm{e}}(X)<\theta+\varepsilon'\}$. This can be done by extracting the
frequent itemsets of $\Ds_\mathrm{e}$ with respect to $\theta$ and partitioning
this collection appropriately into $\mathcal{T}_\mathrm{e}$ and $\mathcal{G}$.
In the second phase, we compute a value $\varepsilon''$ such that the evaluation dataset
$\Ds_\mathrm{v}$ is an $\varepsilon''$-approximation to XXX the range space of
the itemsets in $\mathcal{G}$, with
probability at least $1-\delta_2$. To obtain $\varepsilon''$ using
Thms.~\ref{thm:XXX} and~\ref{thm:XXX}, we need to compute upper bounds to the
VC-dimension and to the empirical VC-dimension of XXX the range space of
the itemsets in $\mathcal{G}$. The d-index {XXX} of
$\Ds_\mathrm{v}$ gives an upper bound to the former but a more refined
technique is needed to compute the latter. We need to solve the SUKP on
$\mathcal{G}$ using the length of the transactions of $\Ds_\mathcal{v}$ {XXX:
need better description, but it will depend on how we organize the paper}.
Once we have obtained $\varepsilon''$, we compute the set
\[
\mathcal{T}_\mathrm{v}=\{X\subseteq\Itm ~:~ X\in\mathcal{G} \mbox{ and }
f_{\Ds_\mathrm{v}}(X)\ge\theta+\varepsilon''\}.\]
The test returns the collection of itemsets $\mathcal{T}_\mathrm{e}\cup\mathcal{T}_\mathrm{v}$.

XXX: Add pseudocode?

\begin{lemma}
  The FWER of the test is at most $\delta$.
\end{lemma}

\begin{proof}
  XXX outline

  Consider the two events $\mathrm{E}_1$=``$\Ds_\mathrm{e}$ is an
  $\varepsilon'$-approximation for XXX the range space of all itemsets'' and $\mathrm{E}_2=$``$\Ds_\mathrm{v}$ is an
  $\varepsilon''$-approximation for XXX the range space of the itemsets in $\mathcal{G}$. From
  the definition of $\varepsilon'$, $\varepsilon''$, $\delta'$ and $\delta''$ we
  know that the event $\mathrm{E}=\mathrm{E}_1\cap\mathrm{E}_2$ occurs with probability at
  least $1-\delta$. Suppose that indeed $\mathrm{E}$ occurs.

  Given that $\mathrm{E}_1$ occurs, $\Ds_\mathrm{e}$ is a
  $\varepsilon'$-approximation for XXX the range space of all itemsets. Then all
  the itemsets with frequency in $\Ds_\mathrm{e}$ at least $\theta+\varepsilon'$
  must have a real frequency at least $\theta$. This equals to say that all
  itemsets in $\mathcal{T}_\mathrm{e}$ are True Frequent Itemsets
  ($\mathcal{T}_\mathrm{e}\subseteq TFI(\theta)$.

  Given that $\mathrm{E}_2$ occurs, then we know that all itemsets in
  $\mathcal{G}$ have frequency in $\Ds_\mathrm{v}$ that is at most
  $\varepsilon''$ far from their real frequency. In particular, this means that
  if they have a true frequency less than $\theta$, then their frequency in
  $\Ds_\mathrm{v}$ is less than $\theta+\varepsilon''$. Hence, all itemsets in
  $\mathcal{G}$ that have frequency at least $\theta+\varepsilon''$ must have
  real frequency at least $\theta$ and therefore be True Frequent Itemsets.
  These are exactly the itemsets in $\mathcal{T}_\mathrm{v}$, so
  $\mathcal{T}_\mathrm{v}\subseteq TFI(\theta)$. 

  We can then conclude that if $\mathrm{E}$ occurs, we have
  $\mathcal{T}_\mathrm{e}\cup \mathcal{T}_\mathrm{v}\subseteq TFI(\theta)$.
  Since $\mathrm{E}$ occurs with probability at least $1-\delta$, this equals to
  say that the probability of the event ``$\exists X\in\mathcal{T}_\mathrm{e}\cup
  \mathcal{T}_\mathrm{v}$ such that $r_p(X)<\theta$'' is at most $\delta$, and
  the thesis follows.
 \end{proof}
\end{document}

