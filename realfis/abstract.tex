% at most 300 words
\begin{abstract} \small\baselineskip=9pt
   Frequent itemsets mining is a fundamental primitive in data mining, requiring to identify all itemsets that appear in a fraction at least $\theta$ of the transactional dataset. However, a transactional dataset only represents a sample from the underlying process that generates the data, the understanding of which is the ultimate goal of data mining. In general, 
   the generative process yields transactions according to a probability distribution.
  Therefore, the output of traditional frequent itemsets mining algorithms can
  and typically
  does contain a large number of spurious patterns that only happen to have a support
  larger than the minimum threshold $\theta$ in the dataset at hand because of
  the stochasticity inherent in the dataset generation process.  In order for the end user to take
  informed decisions using the mining results, it is necessary that the
  returned collection only contains \emph{Real Frequent Itemsets (RFI's)}, i.e.,
  itemsets $A$ such that their \emph{real frequency} (the probability that
  $A$ appear in a transaction) is greater or equal to the the minimum threshold $\theta$.
  In this work we present an algorithm to extract a collection $C$ of RFI's
  while keeping the probability
  that one or more spurious itemsets are included in $C$ to within a
  user-specified limit.  In other words, we present a statistical test for RFI's
  for which we can guarantee that the \emph{Family-Wise Error Rate} is within
  the user-specified limits. We use results from statistical learning theory
  involving the Vapnik-Chervonenkis (VC) dimension of the problem at hand to
  accomplish this goal. This allows us to achieve, on the same data, much
  stricter bounds on the probability of a Type I error than what could be done
  using traditional multiple hypothesis testing corrections. In our experimental
  evaluation we show empirically that our test has very high statistical power,
  i.e., the output collection contains a large fraction of the RFI's.
\end{abstract}

{\bf Categories and Subject Descriptors:} H.2.8 [Database Management]: Database Applications -- \emph{Data Mining}

{\bf Keywords:} Frequent itemsets, Statistical test, VC-dimension, Type-1 error, Multiple
hypothesis testing, Family-wise error rate.

