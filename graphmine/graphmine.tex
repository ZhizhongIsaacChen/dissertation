\chapter{Mining frequent subgraphs through sampling}\label{ch:graphmine}
Very large graphs with millions if not billions of nodes are available today and
are studied by researchers in multiple disciplines.
The complexity inherent in the combinatorial nature of many problems defined on
graphs makes efficient analysis of these huge networks a hard problem. 

In this chapter we will develop an algorithm to extract a collection of frequent
subgraphs up to a specified size from a very large graph. Finding this
collection is a key problem of many applications to a number of fields, including
computational biology, social network analysis, the study of electronic
circuits, and the behavior and structure of software.

\section{Problem definition}\label{sec:graphminesettings}
Let $G=(V,E)$ be a graph which can be either directed or
undirected. For any graph we define the size of the graph as the number of
vertices of the graph. An induced subgraph $G'$ of $G$  is a graph $G'=(V',E')$
such that $V'\subseteq V$ and $E'\subseteq E$ such that if $u,v\in V'$ and
$(u,v)\in E$, then $(u,v)\in E'$. Given two graphs $H=(V_H,E_H)$ and
$K=(V_K,E_K)$, they are said to be isomorphic if it is possible to define a
one-to-one function $f:V_H\rightarrow V_K$ such that $(u,v)\in E_H$ if and only
if $(f(u),f(v))\in E_K$.

Given an integer $k>0$ and a set $\mathcal{S}$ of
non-isomorphic connected subgraphs of size $k$, let $S\in\mathcal{S}$ be a
subgraph and let $N(S)$ be the number of induced subgraphs of $G$ that are
isomorphic to $S$. We define the concentration of $S$ in $G$ with respect to
$\mathcal{S}$ as
\[
C(S,\mathcal{S})=\frac{N(S)}{\sum_{R\in\mathcal{S}}N(R)}.\]
The notation for $N(\cdot)$ and $C(\cdot)$ do not include a specification of the
graph $G$ because this will always be clear from the context and $G$ is
considered fixed. We will drop the specification of the set $\mathcal{S}$ from
the notation of $C(\cdot)$ when the set $\mathcal{S}$ is clear from the context.

Given a collection $\mathcal{C}$ of connected subgraphs, potentially of
different sizes, and a real parameter $\theta\in(0,1)$, we are interested in
finding the collection $\mathsf{FS}(G,\theta,\mathcal{C})$ of frequent subgraphs from
$\mathcal{C}$ that have concentration at least $\theta$ in $G$. Formally,
\[
\mathsf{FS}(G,\theta,\mathcal{C})=\{S\in\mathcal{C} ~:~ C(S,\mathcal{S})\ge\theta\}.\]
We will drop the specification of $\mathcal{C}$ and $G$ from $\mathsf{FS}(\cdot)$ when
it is clear from the context.

Given two graphs $H$ and $K$, finding whether $H$ has an induced subgraph
isomorphic to $K$ is an NP-complete problem, which implies that finding the
collection of frequent subgraphs is also computationally hard.

In this work we are interested in efficiently extract an approximation of the
set of frequent subgraphs, defined as follows:
\begin{definition}\label{def:grapheapprox}
  Given a graph $G$, a collection of connected graphs $\mathcal{C}$, a minimum
  concentration frequency $\theta\in(0,1)$ and an error-controlling parameter
  $\varepsilon\in(0,1)$, an $\varepsilon$-approximation of the set
  $\mathsf{FS}(G,\theta,\mathcal{C})$ is a collection $\mathcal{R}$ of pairs
  $(H,\phi)$ where $H\in\mathcal{C}$ and $\phi\in(0,1)$. The collection
  $\mathcal{R}$ is such that
  \begin{itemize}
    \item For all graphs $H\in\mathcal{C}$ with $c(H)\ge\theta$ there is a pair
      $(H,\phi)$ in $\mathcal{R}$.
    \item $\mathcal{R}$ contains no pair $(H,\phi)$ for any $H$ such that
      $c(H)<\theta-\varepsilon$.
    \item For all pairs $(H,\phi)\in\mathcal{R}$ we have $|c(H)-\phi|\le\varepsilon$.
  \end{itemize}
\end{definition}
Our algorithm will sample connected subgraphs from $G$ uniformly at random and
return an $\varepsilon$-approximation with probability at least $1-\delta$, for
a user-supplied value $\delta\in(0,1)$.

\section{Proposed work}\label{sec:graphimineproposal}
A number of previous works explored the task of efficiently compute the set of
frequent subgraphs. They developed exact methods and heuristics to avoid the
issue of finding subgraph
isomorphisms~\citep{BaskervilleP06,GrochowK07,OmidiSMN09}. Some of them make use
of sampling techniques to speed up the extraction of the frequent subgraphs at
the cost of accepting an approximation, not always well defined, of the
collection~\cite{KashtanIMA04}. Sampling from graphs is a difficult problem by
itself, especially if one is interested in sampling according to a specific
distribution~\citep{LeskovecF06,ZouH10}. The problem has been studied in the
literature and a number of algorithms were introduced to sample according to some
distributions~\citep{LeeXE12}. Some of these works are especially interesting
for us because they deal with the problem of uniformly sampling connected
subgraphs of a specific size from the graph~\citep{Wernicke06,RibeiroS10,LuB12}.
There are however two main issues with the algorithms suggested in these works.
The first issue of these and many other works dealing with sampling of subgraphs
is that there is currently no known method to compute the sample size sufficient
to achieve a well-defined approximation of the collection of frequent subgraphs,
or in other words, sufficient to control all the deviations of the unbiased
estimators of the concentrations of the subgraphs from their expectation, a
condition sufficient to ensure that it is possible to mine a good, parametrized,
and well defined approximation of the collection of frequent subgraphs.  The
second issue is that none of them offers a clear method to extract sample
exactly $n$ connected subgraphs of a fixed size uniformly at random from the set
of all subgraphs of a given size, because in these works, the size of the sample
is itself a random variable and it appears that one can only easily control its
expectation. This is due to the fact that the subgraphs sampled by these
algorithms correspond to the last generation in a branching process that walk
along a tree data structure that allow for the efficient generation of all
subgraphs. 

We want to address both issues in this work, as they both revolve around the
sample size, and indeed they both need a clear solution if we want to obtain an
$\varepsilon$-approximation of the set of frequent subgraphs by extracting
sample subgraphs. For the issue of determining the sample size, we plan to use
the tools offered by statistical learning theory like the VC-dimension to bound
all concentration deviations simultaneously. In order to study how to better
control the size of the sample obtained from the methods presented in the
literature, we will study the theory of branching processes in order to find the
right distribution(s) to control the sampling along the three data structure.

