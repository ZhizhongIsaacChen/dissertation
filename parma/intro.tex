\section{Introduction}\label{sec:intro}
The discovery of (top-K) Frequent Itemsets and Association Rules (FIM)
is a fundamental primitive in data mining and databases applications. The
computational problem is defined in the general setting of a transactional dataset
-- a collection of transactions where each transaction is a set of items.
With datasets increasing both in size and complexity, the computation
for FIM faces scalability challenges in both space and time. 
%
%With the increase in size and complexity of available datasets, the computation
%involved in extracting frequent itemsets, top-K frequent itemsets, and
%association rules is inherently demanding in both
%time and space requirements. 
Datasets have
now reached the tens of terabytes scale and it is no longer reasonable to assume
that such massive amounts of data can be easily processed by a single machine
in a sequential fashion. 
%

A typical exact algorithm scans the entire dataset,
possibly multiple times, and stores intermediate counts of a large number of
possible frequent itemsets candidates~\cite{AgrawalS94,HanPY00}. 
%
The cost of these algorithms can be split in two independent components:
the \emph{scanning} cost and the \emph{mining} cost. The scanning cost includes
all operations that directly handle the transactions in the dataset, and scales
with the \emph{size} of the dataset, i.e., the number of such transactions. Examples
include the scanning of the dataset to build the FP-Tree in
FP-Growth~\cite{HanPY00} or to compute the actual frequencies of candidate
frequent itemsets in APriori~\cite{AgrawalIS93}.  The mining cost refers to the
operations in derived data structures and does not require access to the
dataset. Examples include the operations performed on the FP-Tree once it has
been generated, and the creation of candidate itemsets of length $i+1$ at the
end of phase $i$ in APriori.  This cost scales with the \emph{complexity} of the
dataset, i.e., the number of items, the number and distribution of frequent
itemsets, and the underlying process that generated the transactions. It also
depends on parameters given to the algorithm, such as the desired frequency
threshold.

In this paper we are concerned with the scalability of FIM with respect to
the size of the dataset, or the number of transactions. 
In many practical settings for FIM, the process generating the data changes very
slowly or not at all, especially when compared to the data generation rate,
therefore the number and frequency distribution of the frequent itemsets grows
much slower than the size of the dataset. For example, the number of items
available on the catalog of an e-commerce website grows much slower than the
number of purchases by customers, each of which corresponds to a transaction.
Therefore the scanning component grows faster than the mining one, and soon
becomes dominant. 

We introduce a randomized parallel algorithm for approximate frequent itemset
mining, PARMA, that makes the scanning step of FIM embarassingly parallel, thus
exhibiting near-linear speedup with the number of machines.  PARMA combines
random sampling and parallelization techniques in a novel fashion.  It mines, in
parallel, a set of small random samples and then filters and aggregates the
collections of frequent itemsets or association rules obtained from each sample.
Our work is orthogonal to other approaches, like PFP~\cite{LiWZZC08}, which
focuses on parallelizing the mining phase in order to decrease the corresponding
component of the cost. 
%
Due to the use of random sampling, the output of PARMA is an approximation of
the collection of FI's or AR's in the dataset, but leveraging on previous
work~\cite{RiondatoU11}, PARMA offers tight probabilistic guarantees on the
quality of the approximated collections returned in output. In particular it
guarantees that the output is an $\varepsilon$-approximation of the real
collection with probability at least $1-\delta$, where $\varepsilon$ and
$\delta$ are parameters specified by the user (see Section~\ref{sec:def} for
formal definitions). 
PARMA is designed on
MapReduce~\cite{DeanG08}, a novel parallel/distributed architecture that has
raised significant interest in the research and industry communities. MapReduce
is capable of handling very large datasets and efficiently executing parallel
algorithms like PARMA.

To our knowledge PARMA is the first algorithm to exploit the combination of
random sampling and parallelization for the task of Association Rules Mining.  A
number of previous works explored either parallel
algorithms~\cite{BuehrerPTKS07,CongHHP05,EHZaiane06,FangEtAl08,LiuLZT07,OzkuralUA11,JinYA05,Zaki99}
or random
sampling~\cite{Toivonen96,ZakiPLO97,Parthasarathy02,PietracaprinaRUV10,LiG04,RiondatoU11}
for the FIM task, but the two approaches have been seen somewhat orthogonal
until today. In PARMA, the disadvantages of either approach are evened out by
the advantages of the other. In the spirit of \emph{moving computation to the
data} to minimize communication, we avoid data replication, and preserve the
advantages of parallelization by using of multiple independent small random
samples of the dataset which are mined in parallel and have only their results
aggregated. Similarly, we are not subject to the inherent trade-off between
the size of the random sample and the accuracy of the approximation that can be
obtained from it, as PARMA would only have to mine more samples of the same size
in parallel to get higher quality approximations.

Although PARMA is not the first algorithm to use MapReduce to solve the
Association Rule Mining task, it differs from and enhances previous
works~\cite{CryansRC10,GhotingKPK11,Hammoud11,LiWZZC08,LiZ11,YangLF10,ZhouZCLF10}
in two crucial aspects. First, it significantly reduces the data that is
replicated and transmitted in the \emph{shuffle} phase of MapReduce.  Second,
PARMA is not limited to the extraction of Frequent Itemsets but can also
directly compute the collection of Association Rules in MapReduce. In previous
works, association rules had to be created sequentially after the Frequent
Itemsets had been computed in MapReduce. 

%EVALUATION
We conducted an extensive experimental evaluation to test the relative
performance, scalability and accuracy of PARMA across a wide range of
parameters and datasets. Our results suggest that PARMA can
significantly outperform exact mining solutions, has
near-linear speedup, and, as data and nodes are scaled together, is
able to achieve near constant runtimes. Also, our accuracy evaluation
shows that PARMA consistently computes approximated collections
of higher quality than what can be analytically guaranteed.

In this paper:
\begin{enumerate*}
\item We present PARMA, the first randomized MapReduce algorithm for discovering
  approximate collections of frequent itemsets or association rules with
  near-linear speedup.
\item We provide analytical guarantees for the quality of the approximate
  results generated by the algorithm.
\item We demonstrate the effectiveness of PARMA on many datasets and compare
  the performance of our implementation to that of several exact FIM algorithms
  on MapReduce.
\end{enumerate*}


%We can see the
%cost of a step of a mining algorithm as being accounted to either of the
%scanning or the mining component depending on the nature of the step. If the
%step involves looking, touching, or in any way operating on a transaction in the
%dataset, then it gets accounted to the scanning component of the cost. Examples
%of this type of operation include the scanning of the dataset to build the
%FP-Tree in FP-Growth~\cite{HanPY00} or to compute the actual frequencies of
%candidate frequent itemsets in APriori~\cite{AgrawalIS93}. On the other hand,
%when the algorithm is working on computing (candidate) frequent itemsets or
%their frequencies, then the cost is accounted to the mining component. No access
%to the dataset is needed in these steps. This kind of steps include all the
%operation performed on the FP-Tree once it has been generated, and the creation
%of candidate itemsets of length $i+1$ at the end of phase $i$ in APriori. While
%in APriori operations of either type are heavily interleaved, in FP-Growth there
%is a clear separation between the two: the cost of constructing the FP-Tree is
%accounted to the scanning component, while any successive operation is part of
%the mining component. It is easy to see this distinction, as dataset is no
%longer needed once the FP-Tree has been built. Because of this clear separation,
%in this discussion we will refer to FP-Growth, but a similar reasoning can be
%done for APriori. 
%It is clear that the scanning component of the cost is heavily
%dependent on the number of transactions in the dataset, and independent on the
%other variables of the instance of the problems, like the number of items, the
%number and distribution of frequent itemsets, the frequency threshold, and the
%underlying process that generated the transactions. On the other hand, the
%mining component is independent on the dataset size, and instead depends on all
%the above parameters. In particular, notice that if we keep them constant, and
%just increase the dataset size, the mining component will not change or only
%slightly change. This is clear in FP-Growth, as the number of operations
%performed on the FP-Tree depends on the number and distribution of the
%frequencies of the itemsets, which were fixed, but not on the size of the
%dataset. Because of this, as the dataset grows, the scanning component of the
%cost becomes more and more relevant and dominates the mining component. 

%In
%conclusion, it is of the foremost importance to cut down the scanning cost in
%order to achieve better performances in FIM. This is exactly the main design
%goal of our randomized parallel algorithm for FIM, PARMA, which combines random
%sampling and parallelization techniques in a novel fashion to achieve the goal
%of cutting down the \emph{scanning cost}, and therefore the dependency of the
%runtime from the dataset size. PARMA mines, in parallel, a set of small random
%samples and then filters and aggregates the collections of frequent itemsets or
%association rules obtained from each sample. 
%This is orthogonal to other
%approaches, like PFP~\cite{LiWZZC08} which focused on parallelizing the mining
%phase in order to decrease the corresponding component of the cost. 
%
%In this paper we start from a simple but key
%observation on the nature of the cost of FIM and introduce a parallel algorithm
%which scales very well with the size of the dataset. We argue that the cost of
%mining frequent itemsets and association rules can be split into two independent
%components: the \emph{scanning} cost and the \emph{mining} cost. The former is
%due to the need for any algorithm to scan the transactions in the dataset in
%order to compute frequencies and generate candidate itemsets. It depends on the
%size of dataset but is independent from the process that generated the data,
%i.e., from the actual number of frequent itemsets. An example of scanning cost
%is the time taken by FP-Growth~\cite{HanPY00} to build the FP-Tree, which
%requires two scans of the dataset to compute the frequencies of
%single items. The latter mining cost instead is the cost of actually compute the
%frequent itemsets, and is completely independent on the size of the datasets,
%whose transactions are not used by the algorithms in this phase. This cost
%depends on the number of frequent itemsets, and therefore only on the underlying
%process that generates the data (the study of this process is the final goal of
%association rules mining and of knowledge discovery in general). It is therefore
%independent on the size of the dataset, as long as the process does not change,
%because the number of frequent itemsets at a given frequency threshold will be
%the same. An example of this cost is the extraction of frequent itemsets and
%removal of infrequent ones from the FP-Tree in FP-Growth.

%Our algorithm, named PARMA, combines random sampling and parallelization
%techniques in a novel fashion to achieve the goal of cutting down the
%\emph{scanning cost}, and therefore the dependency of the runtime from the
%dataset size.

%A number of previous works explored either parallel
%algorithms~\cite{BuehrerPTKS07,CongHHP05,EHZaiane06,FangEtAl08,LiuLZT07,OzkuralUA11,JinYA05,Zaki99}
%or random
%sampling~\cite{Toivonen96,ZakiPLO97,Parthasarathy02,PietracaprinaRUV10,LiG04,RiondatoU11}
%for the FIM task, but the two approaches have been seen somewhat orthogonal
%until today. They are both aimed at the same final result, faster FIM, but they
%tackle the issues in two different ways, and offers different advantages, while
%suffering from different disadvantages. In PARMA, we combined random sampling
%and parallelization so that the disadvantages of one approach are evened out by
%the advantages of the other. In parallel algorithms for FIM, the computation and
%the data are distributed across different machines and then the results are
%aggregated. However, these algorithms require large replication and exchange of
%data between the individual machines, significantly reducing the gain offered by
%the parallel computation. In the spirit of \emph{moving computation to the data}
%to minimize communication, we try to avoid data replication as much as possible
%to preserve the advantages of parallelization. This is made possible by our use
%of multiple independent small random samples of the dataset which are mined in
%parallel and only their results are then aggregated. On the other hand, the
%application of random sampling to FIM starts from the observation that
%high-quality approximated results of the collections of FI's and AR's are often
%sufficient for practical applications. There is an inherent trade-off between
%the size of the random sample and the accuracy of the approximation that can be
%obtained from it, but a larger sample requires more time to be mined. While
%these algorithms are designed to obtain the best possible approximation from the
%smallest possible sample, a very high-quality approximation still requires a
%large sample that takes significant computation time and space to process,
%defeating their original goal of sampling. We avoid this roadblock by mining a
%number of small random samples in parallel, and then filter and aggregate the
%results from each sample in a scalable way.

%To speed up the
%computation, a number of works explored parallel and distributed
%algorithms~\cite{BuehrerPTKS07,CongHHP05,EHZaiane06,FangEtAl08,LiuLZT07,OzkuralUA11,JinYA05,Zaki99}.
%The computation and the data are distributed across different machines and then
%the results are aggregated. However, these algorithms require large replication
%and exchange of data between the individual machines, significantly reducing the
%gain offered by the parallel computation.  
%An alternative approach to streamline the computation, orthogonal to
%the one just described, is to settle for an output that closely approximates the exact
%solution, in both the collection of frequent itemsets and their frequency.
%High-quality approximated results are often sufficient for practical
%applications. Recent works presented algorithms that mine only a random subset
%of the dataset to obtain high-quality approximations of the collections of
%Frequent Itemsets and Association
%Rules~\cite{Toivonen96,ZakiPLO97,Parthasarathy02,PietracaprinaRUV10,LiG04,RiondatoU11}.
%There is an inherent trade-off between the size of the random sample and the
%accuracy of the approximation that can be obtained from it, but a larger sample
%requires more time to be mined. While these algorithms are designed to obtain
%the best possible approximation from the smallest possible sample, a
%high-quality approximation still requires a large sample that takes significant
%computation time and space to process.  
%In this paper we present PARMA, a randomized parallel distributed algorithm to
%extract approximations of the collections of frequent itemsets and association
%rules from large datasets. PARMA combines random sampling and parallel
%distributed processing  to speed up the mining phase. {\bf XXX: How?}
%To our knowledge PARMA is the first algorithm to exploit the combination of
%random sampling and parallelization for the task of Association Rules Mining.
%Due to the use of random sampling, the output of PARMA is an approximation of
%the collection of FI's or AR's in the dataset, but leveraging on previous
%work~\cite{RiondatoU11}, PARMA offers tight probabilistic guarantees on the
%quality of the approximated collections returned in output. In particular it
%guarantees that the output is an $\varepsilon$-approximation of the real
%collection with probability at least $1-\delta$, where $\varepsilon$ and
%$\delta$ are parameters specified by the user (see Section~\ref{sec:def} for
%formal definitions). We conducted an extensive experimental evaluation of PARMA
%and our results suggest that it consistently computes approximated collections
%of higher quality than what can be guaranteed by the analysis. A comparison of
%PARMA with exact parallel algorithms for Association Rules Mining shows that
%PARMA consistently has significant performance gains.
%PARMA is designed on MapReduce~\cite{DeanG08}, a novel parallel/distributed
%architecture that has raised significant interest in the research and industry
%communities. MapReduce is capable of handling very large datasets and
%efficiently executing parallel algorithms like PARMA. 
%

