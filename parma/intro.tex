The discovery of (top-K) Frequent Itemsets and Association Rules (FIM)
is a fundamental primitive in data mining and databases applications. The
computational problem is defined in the general setting of a transactional dataset
-- a collection of transactions where each transaction is a set of items.
With datasets increasing both in size and complexity, the computation
for FIM faces scalability challenges in both space and time. Datasets have
now reached the tens of terabytes scale and it is no longer reasonable to assume
that such massive amounts of data can be easily processed by a single machine
in a sequential fashion. 

A typical exact algorithm scans the entire dataset,
possibly multiple times, and stores intermediate counts of a large number of
possible frequent itemsets candidates~\citep{AgrawalS94,HanPY00}. The cost of
these algorithms can be split in two independent components:
the \emph{scanning} cost and the \emph{mining} cost. The scanning cost includes
all operations that directly handle the transactions in the dataset, and scales
with the \emph{size} of the dataset, i.e., the number of such transactions. Examples
include the scanning of the dataset to build the FP-Tree in
FP-Growth~\citep{HanPY00} or to compute the actual frequencies of candidate
frequent itemsets in APriori~\citep{AgrawalIS93}.  The mining cost refers to the
operations in derived data structures and does not require access to the
dataset. Examples include the operations performed on the FP-Tree once it has
been generated, and the creation of candidate itemsets of length $i+1$ at the
end of phase $i$ in APriori.  This cost scales with the \emph{complexity} of the
dataset, i.e., the number of items, the number and distribution of frequent
itemsets, and the underlying process that generated the transactions. It also
depends on parameters given to the algorithm, such as the desired frequency
threshold.

In this paper we are concerned with the scalability of FIM with respect to
the size of the dataset, or the number of transactions. 
In many practical settings for FIM, the process generating the data changes very
slowly or not at all, especially when compared to the data generation rate,
therefore the number and frequency distribution of the frequent itemsets grows
much slower than the size of the dataset. For example, the number of items
available on the catalog of an e-commerce website grows much slower than the
number of purchases by customers, each of which corresponds to a transaction.
Therefore the scanning component grows faster than the mining one, and soon
becomes dominant. 

We introduce a randomized parallel algorithm for approximate frequent itemset
mining, PARMA, that makes the scanning step of FIM embarassingly parallel, thus
exhibiting near-linear speedup with the number of machines. PARMA combines
random sampling and parallelization techniques in a novel fashion.  It mines, in
parallel, a set of small random samples and then filters and aggregates the
collections of frequent itemsets or association rules obtained from each sample.
Our work is orthogonal to other approaches, like PFP~\citep{LiWZZC08}, which
focuses on parallelizing the mining phase in order to decrease the corresponding
component of the cost. Due to the use of random sampling, the output of PARMA is
an approximation of
the collection of FIs or ARs in the dataset, but leveraging on previous
work~\citep{RiondatoU12}, PARMA offers tight probabilistic guarantees on the
quality of the approximated collections returned in output. In particular it
guarantees that the output is an $\varepsilon$-approximation of the real
collection with probability at least $1-\delta$, where $\varepsilon$ and
$\delta$ are parameters specified by the user (see Section~\ref{sec:parmadef} for
formal definitions). 
PARMA is designed on
MapReduce~\citep{DeanG08}, a novel parallel/distributed architecture that has
raised significant interest in the research and industry communities. MapReduce
is capable of handling very large datasets and efficiently executing parallel
algorithms like PARMA.

To our knowledge PARMA is the first algorithm to exploit the combination of
random sampling and parallelization for the task of Association Rules Mining. 

A number of previous works explored either parallel
algorithms~\citep{BuehrerPTKS07,CongHHP05,EHZaiane06,FangEtAl08,LiuLZT07,OzkuralUA11,JinYA05,Zaki99}
or random
sampling~\citep{Toivonen96,ZakiPLO97,Parthasarathy02,PietracaprinaRUV10,LiG04,RiondatoU12}
for the FIM task, but the two approaches have been seen somewhat orthogonal
until today. In PARMA, the disadvantages of either approach are evened out by
the advantages of the other. In the spirit of \emph{moving computation to the
data} to minimize communication, we avoid data replication, and preserve the
advantages of parallelization by using of multiple independent small random
samples of the dataset which are mined in parallel and have only their results
aggregated. Similarly, we are not subject to the inherent trade-off between the
size of the random sample and the accuracy of the approximation that can be
obtained from it, as PARMA would only have to mine more samples of the same size
in parallel to get higher quality approximations.

Although PARMA is not the first algorithm to use MapReduce to solve the
Association Rule Mining task, it differs from and enhances previous
works~\citep{CryansRC10,GhotingKPK11,Hammoud11,LiWZZC08,LiZ11,YangLF10,ZhouZCLF10}
in two crucial aspects. First, it significantly reduces the data that is
replicated and transmitted in the \emph{shuffle} phase of MapReduce.  Second,
PARMA is not limited to the extraction of Frequent Itemsets but can also
directly compute the collection of Association Rules in MapReduce. In previous
works, association rules had to be created sequentially after the Frequent
Itemsets had been computed in MapReduce. 

We conducted an extensive experimental evaluation to test the relative
performance, scalability and accuracy of PARMA across a wide range of
parameters and datasets. Our results suggest that PARMA can
significantly outperform exact mining solutions, has
near-linear speedup, and, as data and nodes are scaled together, is
able to achieve near constant runtimes. Also, our accuracy evaluation
shows that PARMA consistently computes approximated collections
of higher quality than what can be analytically guaranteed.

In this chapter:
\begin{enumerate}
\item We present PARMA, the first randomized MapReduce algorithm for discovering
  approximate collections of frequent itemsets or association rules with
  near-linear speedup.
\item We provide analytical guarantees for the quality of the approximate
  results generated by the algorithm.
\item We demonstrate the effectiveness of PARMA on many datasets and compare
  the performance of our implementation to that of several exact FIM algorithms
  on MapReduce.
\end{enumerate}

