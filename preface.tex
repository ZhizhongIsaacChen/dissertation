\begin{quote}
  {\em ``Dedique cor meum ut scirem prudentiam atque doctrinam erroresque et
  stultitiam et agnovi quod in his quoque esset labor et adflictio spiritus. Eo
  quod in multa sapientia multa sit indignatio et qui addit scientiam addat et
  laborem.''}\footnote{``And I applied my heart to know wisdom, and to know madness
  and folly: I perceived that this also was a striving after wind. For in much
  wisdom is much grief; and he that increaseth knowledge increaseth sorrow.'' 
  (American Standard Version)} Ecclesiates, 1:17-18.
\end{quote}

This dissertation is the work of many people.

I am deply indebted to my advisor Eli Upfal. We met in Padova in October 2007
and after two weeks I asked him whether I could come to the States to do my
master thesis. Surprisingly for me, he said yes. Since then he helped me in all
possible ways, even when I seemed to run away from his help (and perhaps I
foolishly was). This dissertation started when he casually suggested
VC-dimension as a tool to overcome some limitations of the ``Chernoff + Union
bound'' technique. After that, Eli was patient to let me find out what it was,
interiorize it, and start using it. He was patient when I kept using it despite
his suggestions to move on (and I am not done with it yet). He was patient when
it became clear that I no longer knew what I wanted to do after grad school. His
office door was always open for me and I abused of his time and patience. He has
been my mentor and I will always be proud of having been his pupil. I hope he
will be, if not proud, at least not dissatisfied with what I am and will become,
no matter where life leads me.  

U\v{g}ur \c{C}etintemel and Basilis Gidas were on on my thesis committee and have
been incredibly valuable in teaching me about databases and statistics
(respectively) and above all in encouraging me when I had doubts about my future
perspectives.

Fabio Vandin was, in my opinion, my older brother in academia. We came to
Providence almost together from Padua and we are leaving it almost together.
Since I first met him, he was the example that I wanted to follow. If Fabio was
my brother, Olya Ohrimenko was my sister at Brown, with all the up and downsides
of being the older sister. I am grateful to her.

Andrea Pietracaprina and Geppino Pucci were my mentors in Padua and beyond. I
would be ungrateful if I forgot to thank them for all the help, the support, the
encouragement, and the frank conversations that we had on both sides of the
ocean.

Aris Anagnostopoulos and Luca Becchetti hosted me in Rome when I visited
Sapienza in Summer 2011. They taught me more than they probably realize about
doing research. 

Francesco Bonchi was my mentor at Yahoo Labs in Barcelona in the Summer 2012. He
was much more than a boss. He was a colleague in research, a foosball fierce
opponent, a companion of cervezitas on the Rabla de Poble Nou.  He shattered
many cancrenous ``certainties'' that I created for myself, and showed me what it
means to be a great researcher and manager of people.

The Italian gang at Brown was always supportive, encouraging, and provided
moments of fun, food, and festivity:  Michela Ronzani, Marco Donato, Bernardo
Palazzi, Cristina Serverius, Nicole Gerke, Erica Moretti, and many many others.
I will cherish the moments I spent with with them.

Every time I went back to Italy my friends made it feel like I oly
left for a day: Giacomo, Mario, Fabrizio, Nicol\'o, the various Giovanni,
Valeria, Iccio, Matteo, Simone, Elena, and all the others. 

Mackenzie has been amazing in the last year and I believe she will be
amazing for years to come. She supported me whenever I had a doubt or changed
my mind. Whenever I was stressed she made me laugh. Whenever I felt lost, she
made me realize that I was following the path. I hope I gave her something,
because she gave me a lot.

Mamma, Pap\`a, and Giulia have been closest to me despite the oceanic distance.
They were the lighthouse and safe harbour in the nights of doubts, and the solid
rock when I was falling, and the warm fire when I was cold. I will come back,
one day.

This dissertation is dedicated to my grandparents Luciana, Maria, Ezio, and
Guido. I wish one day I could be half of what they are and have been.

