\documentclass{article}
\usepackage{lmodern}
\usepackage[T1]{fontenc} 
\usepackage{textcomp}
\usepackage{dsfont}
\usepackage{fullpage}
\usepackage{amssymb,amsmath,amsthm}
\usepackage{datetime}
\usepackage{natbib}
\newtheorem{lemma}{Lemma}
\def\betw{\mathsf{b}}
\def\XXX{{\bf XXX}}
\def\MR{{\bf MR}}
\def\EK{{\bf EK}}
\def\exp{\mathbb{E}}
\def\var{\mathrm{Var}}
\def\mse{\mathrm{MSE}}
\title{}
\author{}
\begin{document}
%\maketitle
{\bf \large \today, \currenttime}

\section{Tighter Analysis for the BP Algorithm}
The analysis of Algorithm 1 allow us to obtain a tighter analysis for the
algorithm by~\citet{BrandesP07}  (which we denote as \textsf{BP}).

\begin{lemma}\label{lem:variance}
  The estimator $\tilde\betw_\mathsf{BP}(w)$ is \emph{uniformly better} than the
  estimator $\tilde\betw_\mathrm{A1}(w)$. I.e., for any value of $\betw(w)$, we
  have
  \[
  \var[\tilde\betw_\mathsf{BP}(w)]\le\var[\tilde\betw_\mathrm{A1}(w)]\enspace.
  \]
\end{lemma}

\begin{proof}
  Consider the quantity
  \[
  \var[\tilde\betw_\mathrm{A1}(w)]-\var[\tilde\betw_\mathsf{BP}(w)]\enspace.\]
  We will show that the above quantity is greater than or equal to 0, proving
  the thesis.
  Since
  $\exp[\tilde\betw_\mathrm{A1}(w)]=\exp[\tilde\betw_\mathsf{BF}(w)]=\betw(w)$
  and using the definition of variance, we only need to show 
  \begin{equation}\label{eq:centrsamplinequal}
    \exp[(\tilde\betw_\mathrm{A1}(w))^2]-\exp[(\tilde\betw_\mathsf{BF}(w))^2]\ge
    0\enspace.
  \end{equation}

  Let start from computing $\exp[(\tilde\betw_\mathsf{BF}(w))^2]$. For every
  vertex $v$, we define $\alpha_v$ as
  \[
  \alpha_v=\frac{1}{n-1}\sum_{\substack{u\in V
  \\u\neq
  v}}\sum_{p\in\mathcal{S}_{vu}}\frac{\mathds{1}_{\mathsf{Int}(p)}(w)}{\sigma_{vu}}\enspace.
  \]
  Note that 
  \begin{equation}\label{eq:centrsamplbetwalpha}
    \betw(w)=\frac{1}{n}\sum_{v\in V}\alpha(v)\enspace.
  \end{equation}
  Consider, for $1\le i \le r$, the contribution $X_i$ to $\tilde\betw_\mathsf{BP}(w)$
  of the paths computed from the $i$\textsuperscript{th} sampled vertex. $X_i$
  is a random variable that takes value $\alpha_v$ with probability $1/n$, for
  all $v\in V$. We have
  \begin{equation}\label{eq:centrsamplxiexpvar}
    \exp[X_i]=\sum_{v\in V}\frac{1}{n}\alpha_v=\betw(w) \mbox{ and }
    \exp[X_i^2]=\frac{1}{n}\sum_{v\in V}\alpha_v^2, \forall 1\le i\le r\enspace.
  \end{equation}
  Clearly
  \[
  \tilde\betw_\mathsf{BP}(w)=\frac{1}{r}\sum_{i=1}^r X_i\enspace.
  \]
  The variables $X_i$'s are independent and identically distributed so
  \begin{align}\label{eq:centrsamplexpxisquared}
    \exp[(\tilde\betw_\mathsf{BP}(w))^2]&=\frac{1}{r^2}\exp\left[\left(\sum_{i=1}^rX_i\right)^2\right]=\frac{1}{r^2}\sum_{i=1}^r\left(\exp[X_i^2]+\sum_{j=i+1}^r(\exp[X_i]\exp[X_j])\right)=\frac{1}{r}\frac{1}{n}\sum_{v\in
    V}\alpha_v^2+\frac{r-1}{r}(\betw(w))^2,
\end{align}
where we used~\eqref{eq:centrsamplxiexpvar}.

We now compute $\exp[(\tilde\betw_\mathrm{A1}(w))^2]$. 
Consider the contribution $Y_i$ to $\tilde\betw_\mathrm{A1}(w)$ of the
$i$\textsuperscript{th} sampled path by Algorithm 1, for $1\le i\le r$. $Y_i$ is a random variable that takes value
$\mathds{1}_{\mathsf{Int}(p)}(w)$ with probability $\pi(p)$, for every shortest
path $p\in\mathbb{S}_G$. We have
\begin{equation}\label{eq:centrsamplyiexpvar}
  \exp[Y_i]=\exp[Y_i^2]=\betw(w), \forall 1\le i \le r\enspace.
\end{equation}
By definition, 
\[
\tilde\betw_\mathrm{A1}(w)=\frac{1}{r}\sum_{i=1}^rY_i\enspace.
\]
The random variables $Y_i$ are independent and identically distributed so
\begin{align}\label{eq:centrsamplexpyisquared}
  \exp[(\tilde\betw_\mathrm{A1}(w))^2]&=\frac{1}{r^2}\exp\left[\left(\sum_{i=1}^r
  Y_i\right)^2\right]=\frac{1}{r^2}\sum_{i=1}^r\left(\exp[Y_i^2]+\sum_{j=i+1}^r\exp[Y_i]\exp[Y_j]\right)=\frac{1}{r}\betw(w)+\frac{r-1}{r}(\betw(w))^2,
\end{align}
where we used~\eqref{eq:centrsamplyiexpvar}.

We can now rewrite the left side of~\eqref{eq:centrsamplinequal} using
~\eqref{eq:centrsamplbetwalpha},~\eqref{eq:centrsamplexpxisquared},~and~\eqref{eq:centrsamplexpyisquared}:
\begin{align*}
  \exp[(\tilde\betw_\mathrm{A1}(w))^2]-\exp[(\tilde\betw_\mathsf{BF}(w))^2] &=
  \frac{1}{r}\betw(w)+\frac{r-1}{r}(\betw(w))^2 - \frac{1}{r}\frac{1}{n}\sum_{v\in
    V}\alpha_v^2
   -\frac{r-1}{r}(\betw(w))^2 =
    \frac{1}{r}\frac{1}{n}\sum_{v\in V}(\alpha_v - \alpha_v^2)\enspace. 
\end{align*}
Since $\alpha_v\in[0,1]$, we have $\alpha_v-\alpha_v^2\ge 0$ for all $v$, and our proof
is complete.
\end{proof}

The following lemma is an easy consequence of the above.

\begin{lemma}\label{lem:MSE}
  \textsf{BP} has lower expected Mean Squared Error than Algorithm~1:
  \[
  \exp[\mse_\mathsf{BP}]\le \exp[\mse_\mathrm{A1}]\enspace.
  \]
\end{lemma}

\begin{proof}
  We have
  \[ 
  \exp[\mse_\mathsf{BP}] = \exp\left[\frac{1}{n}\sum_{v\in
  V}\left(\tilde\betw_\mathsf{BP}(w)-\betw(v)\right)^2\right]=\frac{1}{n}\sum_{v\in
  V}\exp\left[\left(\tilde\betw_\mathsf{BP}(w)-\betw(v)\right)^2\right]=\frac{1}{n}\sum_{v\in
  V}\var[\tilde\betw_\mathsf{BP}(w)],
  \]
  where we used the linearity of expectation and the fact that the estimator
  $\tilde\betw_\mathsf{BP}(w)$ is unbiased for $\betw(w)$ ($\exp[\tilde\betw_\mathsf{BP}(w)]=\betw(w)]$).
  Analogously for Algorithm~1:
  \[
  \exp[\mse_\mathrm{A1}] = \frac{1}{n}\sum_{v\in V}\var[\tilde\betw_\mathrm{A1}(w)]\enspace.
  \]
  Hence
  \[
  \exp[\mse_\mathrm{A1}]-\exp[\mse_\mathsf{BP}]=\frac{1}{n}\sum_{v\in
  V}\left(\var[\tilde\betw_\mathrm{A1}(w)]-\var[\tilde\betw_\mathsf{BP}(w)]\right),
  \]
  and from Lemma~\ref{lem:variance} we have that each addend of the sum is
  non-negative and so is the above expression, concluding our proof.
\end{proof}


\begin{lemma}
  Let $G=(V,E)$ be a graph, and $\varepsilon,\delta\in(0,1)$. Let
  \[
  r =
  \frac{c}{\varepsilon^2}\left(\lfloor\log_2(\mathsf{VD}(G)-2\rfloor+1+\ln\frac{1}{\delta}\right)\enspace.
  \]
  and let $S$ be a set of vertices of $G$ sampled uniformly at random with
  replacement from $V$. Let $\tilde\betw_{\mathsf{BP}}(w)$ be the estimated
  value for $\betw(w)$ computed by \textsf{BP} using $S$. Then
  \[
  \Pr(\exists w\in V \mbox{ s.t. }
  |\tilde\betw_{\mathsf{BP}}(w)-\betw(w)|>\varepsilon)<\delta\enspace.
  \]
  %If Algorithm 1 computes an $(\varepsilon,\delta)$-approximation with $r$
  %samples, then the algorithm by~\citet{BrandesP07} does the same with the same
  %number of samples.
\end{lemma}

\begin{proof}
  Consider a simple coupling of the executions of Algorithm 1 and \textsf{BP}
  for $r$ samples: when Algorithm 1 samples a shortest path from a vertex $v$ to a vertex $u$,
  \textsf{BP} samples the vertex $v$. 

Consider an execution of the algorithms and let $X_v$ be the number of times
that vertex $v$ is the starting point of a sampled path. 

The estimation for the betweenness of vertex $w$ computed by \textsf{BP} is
\[
\tilde\betw_{\mathsf{BP}}(w)=\frac{1}{r}\sum_{v\in V}
X_v\underbrace{\left(\frac{1}{n-1}\sum_{\substack{u\in V \\u\neq
v}}\sum_{p\in\mathcal{S}_{vu}}\frac{\mathds{1}_{\mathsf{Int}(p)}(w)}{|\mathcal{S}_{vu}|}\right)}_{Y_v(w)}\enspace.
\]
The estimation for the betweenness of vertex $w$ computed by Algorithm 1
is
\[
\tilde\betw_{\mathrm{A1}}(w)=\frac{1}{r}Z_w,
\]
where $Z_w$ is the number of times that a path containing the vertex $w$ gets
sampled. Now, for each $v$ and for $1\le i\le X_v$, let $p_v^{(i)}$ be the path
sampled the $i$\textsuperscript{th} time that the sampled path had $v$ as
starting vertex, and let $A_v^{(i)}(w)=1$ if $w\in\mathsf{Int}(p_v^{(i)}$, and
$A_v^{(i)}(w)=0$ otherwise. Then it is clear that 
\[
\tilde\betw_{\mathrm{A1}}(w)=\frac{1}{r}\sum_{v\in V}
X_v\frac{\sum_{i=1}^{X_v}A^{(i)}_v(w)}{X_v}\enspace.
\]
We claim that
\[
\mathbb{E}\left[\left.\frac{\sum_{i=1}^{X_v}A^{(i)}_v(w)}{X_v}\right|X_v\right]=Y_v(w),\]
where the expectation is taken over the choice of the paths, which are sampled
according to the distribution $\pi$.
To see this, consider $\mathbb{E}[A_v^{(i)}(w) ~|~ X_v]$. It
is clear that $A^{(i)}(w)$ and $X_v$ are independent, for $1\le i\le X_v$. From
this and the definition of expectation we have 
\[
\mathbb{E}\left[\left.A_v^{(i)}(w)\right|X_v\right] =
\mathbb{E}\left[A_v^{(i)}(w)\right]=\frac{1}{n-1}\sum_{\substack{u\in V
\\u\neq
v}}\sum_{p\in\mathcal{S}_{vu}}\frac{\mathds{1}_{\mathsf{Int}(p)}(w)}{|\mathcal{S}_{vu}|}
=Y_v(w)\enspace.
\]
This expression does not depend on $i$.

The claim then follows easily from the linearity of expectation:
\[
\mathbb{E}\left[\left.\frac{\sum_{i=1}^{X_v}A^{(i)}_v(w)}{X_v}\right|X_v\right] =
\sum_{i=1}^{X_v}\mathbb{E}\left[\left.\frac{A^{(i)}_v(w)}{X_v}\right|X_v\right]=\sum_{i=1}^{X_v}\frac{1}{X_v}\mathbb{E}\left[A_v^{(i)}(w)\right]=\mathbb{E}\left[A_v^{(1)}(w)\right]=Y_v(w)\enspace.
\]
Recall from Lemma~\ref{lem:variance} that the estimator
$\tilde\betw_{\mathsf{BP}}(w)$ has lower variance than the estimator
$\tilde\betw_{\mathrm{A1}}(w)$, for all vertices $w$. 

\XXX: The following is not actually true \MR. Since both estimators have
the same expectation $\betw(w)$, this means that for any $a\in[0,1]$ we have
\[
\Pr\left(|\tilde\betw_{\mathsf{BP}}(w)-\betw(w)|>a\right)\le\Pr(\left(|\tilde\betw_{\mathrm{A1}}(w)-\betw(w)|>a\right)\enspace.
\]

This means that for every collection of $r$ samples for which Algorithm 1 computes estimates
$\tilde\betw_\mathrm{A1}(w)$ such that
\begin{equation}\label{eq:centrsampledapprox}
|\tilde\betw_\mathrm{A1}(w)-\betw(w)|\le\varepsilon, \forall w\in W,
\end{equation}
so would \textsf{BP} using the $r$ samples according to the coupling. For
Algorithm $1$ the event in~\eqref{eq:centrsampledapprox} occurs with probability at least
$1-\delta$ over the choice of samples, and so would be for \textsf{BP},
concluding our proof.
\end{proof}

The estimator $\tilde\betw_{\mathsf{BP}}(w)$ has lower variance than
$\tilde\betw_{\mathrm{A1}}(w)$, which means that it converges faster to the
expectation $\betw(w)$, achieving at least the same guarantees with the same
number of samples. Before always opting for \textsf{BP} (or for the algorithm
by~\citet{GeisbergerSS08}, whose estimators have even lower variance than those
of \textsf{BP}), one should nevertheless take
into account the fact that, per sample, the
computation of $\tilde\betw_{\mathsf{BP}}(w)$ requires more time than the one
for $\tilde\betw_{\mathrm{A1}}(w)$. The difference is even larger if Algorithm 1
uses bidirectional search~\citep{KaindlK97,Pohl69}. We can then see that
Algorithm 1, \textsf{BP}, and the algorithm
by~\citet{GeisbergerSS08} share the same design
principles, but choose different trade-offs between accuracy and speed.  

\MR: my intuition is that we can actually analyze \textsf{BP} using the
VC-dimension and showing that their sampling is ergodic. Recent results show
that the VC-dimension can be used when the sampling process is ergodic~\citep{AdamsN10}.



\bibliographystyle{abbrvnat}
\bibliography{bidirectionalsearch,diamapprox,vcmine,vcgraph,centrality,various}

\end{document}

