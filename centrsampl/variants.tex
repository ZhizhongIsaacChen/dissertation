\ifproof
%\section{Edge Betweenness and Variants of Vertex Betweenness}\label{sec:centrsamplvariants}
\section{Variants of betweenness centrality}\label{sec:centrsamplvariants}
In this section we discuss how to extends our results to some variants of
betweenness centrality.

\subsection{$k$-bounded-distance betweenness}
A ``local'' variant of betweenness, called \emph{$k$-bounded-distance
betweenness}\footnote{Bounded-distance betweenness is also known as
$k$-betweenness. We prefer the former denomination to avoid confusion with
$k$-path betweenness from~\citep{KourtellisASIT12}.} only considers
the contribution of shortest paths of size up to $k+1$~\citep{BorgattiE06,Brandes08}.
For $k>1$ and any pair of distinct vertices $u,v\in V$, $u\neq V$, let
$\mathcal{S}^{(k)}_{uv}\subseteq\mathcal{S}_{uv}$ be the set of shortest paths
from $u$ to $v$ of size at most $k+1$, with
$\sigma^{(k)}_{uv}=|\mathcal{S}^{(k)}_{uv}|$, and let $\mathbb{S}^{(k)}_G$ be the
union of all the $\mathcal{S}^{(k)}_{uv}$. Let
$\mathcal{T}^{(k)}_v\subseteq\mathcal{T}_v$ be the set of all shortest paths
\emph{of size up to $k$} that $v$ is internal to, for each $v\in V$.

%\begin{definition}[\citep{BorgattiE06,Brandes08}]\label{def:centrsamplkboundbetweenness}
\begin{definition}\label{def:centrsamplkboundbetweenness}
  \citep{BorgattiE06,Brandes08} Given a graph $G=(V,E)$ and an integer $k>1$,
  the \emph{$k$-bounded-distance betweenness centrality of a vertex $v\in V$} is
  defined as
  \[
  %\betw(v)=\sum_{(u,w)\in V\times
  %V}\sum_{p\in\mathcal{S}_{uw}}\frac{\mathds{1}_{\mathcal{T}_v}(p)}{|\mathcal{S}_{uw}|}\enspace.
  %\betw(v)=\sum_{p_{uw}\in\mathbb{S}_G}\frac{\mathds{1}_{\mathcal{T}_v}(p)}{|\mathcal{S}_{uw}|}\enspace.
  \kboundbetw^{(k)}(v)=\frac{1}{n(n-1)}\sum_{p_{uw}\in\mathbb{S}^{(k)}_G}\frac{\mathds{1}_{\mathcal{T}^{(k)}_v}(p)}{\sigma^{(k)}_{uw}}
  %=\frac{1}{n(n-1)}\sum_{p_{uw}\in\mathcal{T}^{(k)}_v}\frac{1}{\left|\mathcal{S}^{(k)}_{uw}\right|}
  \enspace.
  \]
\end{definition}

For the case of $k$-bounded-distance betweenness, if we let
$\range_G^{(k)}=\{\mathcal{T}_v^{(k)}~:~ v\in V\}$, it is easy to bound
$\VC\left((\mathbb{S}^{(k)}_G,\range_G^{(k)})\right)$ following the same reasoning as in
Lemma~\ref{lem:vcdimuppbound}.
\begin{lemma}\label{lem:vcdimuppboundk}
$\VC\left((\mathbb{S}^{(k)}_G,\range_G^{(k)})\right)\le\lfloor\log_2(k-1)\rfloor+1$.
\end{lemma}

Given this result, the sample size on line~\ref{algline:forloop}
of Alg.~\ref{alg:algorithm} can be reduced to 
\[ 
  r= \frac{c}{\varepsilon^2}\left(\lfloor\log_2(k-1)\rfloor + 1 +\ln\frac{1}{\delta}\right)
\]
and the computation of the shortest paths on line~\ref{algline:shortestpaths}
can be stopped after we reached the vertices that are $k$ ``hops'' far from $u$.

\subsection{$\alpha$-weighted betweenness}
\citet{OpsahlAS10} defined a parametric variant of betweenness centrality for
weighted networks that can be seen as a generalization of the classical 
definition. The goal behind this new definition is to give the analyst the
possibility of ``penalizing'' shortest paths with many vertices, given the
intuition that there is a cost to be paid to have a message go through a vertex.
The core of the new definition is a different weight function of a path
$p_{uv}=(w_1=u,w_2,\dotsc,w_{|p_{uv}|}=v)$ between two vertices $(u,v)$,
parametrized by a real parameter $\alpha\ge 0$:
\begin{equation}\label{eq:centrsamplgeneralizeddistance}
d_{\alpha,uv}=\sum_{i=1}^{|p_{uv}|-1}(\mathsf{w}((w_i,w_{i+1})))^\alpha\enspace. 
\end{equation}
Clearly for $\alpha=0$, the weights are considered all equal to $1$, and for
$\alpha=1$, the definition falls back to $d_{uv}$. For $0<\alpha<1$, paths with
fewer vertices and higher weights are favored over longer paths with lighter
edges\citep{OpsahlAS10}. The definition of shortest path distance, shortest
path, and of betweenness centrality follow as before
from~\eqref{eq:centrsamplgeneralizeddistance}. Dijkstra's shortest paths algorithm can be
adapted to compute shortest paths according to the distance function
in~\eqref{eq:centrsamplgeneralizeddistance}. To avoid confusion, we call the betweenness
centrality computed using the shortest paths according to the distance function
in~\eqref{eq:centrsamplgeneralizeddistance}, the \emph{$\alpha$-weighted betweenness} and
denote it with $\betw_\alpha(w)$. 

For $\alpha$-weighted betweenness we have that the VC-dimension of the range space
defined on the shortest paths according to the distance
from~\eqref{eq:centrsamplgeneralizeddistance} is still bounded by
$\lfloor\log_2(\VD(G)-1)\rfloor+1$. Clearly, the vertex-diameter $\VD(G)$ is now
defined as the maximum number of vertices in a shortest path according
to~\eqref{eq:centrsamplgeneralizeddistance}.

No special arrangements are needed to obtain an approximation of
$\alpha$-weighted betweenness for all vertices. The sample size is the same as in~\eqref{eq:centrsamplsamplesize}.

\subsection{$k$-path betweenness}
Another interesting variant of betweenness centrality does not consider
\emph{shortest} paths, but rather \emph{simple random walks} of size up to
$k+1$, expressing the intuition that information in a network does not
necessarily spread across shortest paths but has a high probability of ``fading
out'' after having touched at most a fixed number of vertices.

\begin{definition}[\citep{KourtellisASIT12}]\label{def:centrsamplkpathbetweenness}
 Given a graph $G=(V,E)$ and a positive integer $k$, the \emph{$k$-path
 centrality} $\kpathbetw^{(k)}(v)$ of $v\in V$ is defined as the average\footnote{We take the
 average, rather than the sum, as defined in the original work
 by~\citet{KourtellisASIT12}, to normalize the value so that it belongs to the interval
 $[0,1]$. This has no consequences on the results.}, over all possible source
 vertices $s$, of the probability that a simple random walk originating from
 $s$ and extinguishing after having touched (at most) $k+1$ vertices ($s$
 included) goes through $v$.
\end{definition}

We can define another range space
$(\mathbb{S}^{\mathrm{p},k}_G,\range_G^{\mathrm{p},k})$ if we are interested in
$k$-path betweenness. The domain $\mathbb{S}^{\mathrm{p},k}_G$ of the range
space now is the set of all simple random walks starting from any vertex of and
of size up to $k+1$. For each vertex $v\in V$, the range $R_v$ is the subset of
$\mathbb{S}^{\mathrm{p},k}_G$ containing all and only the random walks from
$\mathbb{S}^{\mathrm{p},k}_G$ that have $v$ as internal vertex. It is easy to
see that $\VC\left(\mathbb{S}^{\mathrm{p},k}_G,\range_G^{\mathrm{p},k})\right)$
is at most $\lfloor\log_2(k-1)+1$ following the same reasoning as in
Lemma~\ref{lem:vcdimuppbound} and Lemma~\ref{lem:vcdimuppboundk}.

For the case of $k$-path betweenness, the algorithm is slightly different:
for 
\[
  r= \frac{c}{\varepsilon^2}\left(\lfloor\log_2(k-1)\rfloor + 1 +\ln\frac{1}{\delta}\right)
\]
iterations, rather than sampling a pair of vertices $(uv)$, computing
$\mathcal{S}_{uv}$, and then sampling a shortest path between them, we first sample a
single vertex and an integer $\ell$ uniformly at random from $[1,k+1]$. We then
sample a simple random walk touching $\ell$ vertices, while updating the estimated
betweenness of each touched vertex by adding $1/r$ to its estimation. The method
to sample a simple random walk is described in~\citep{KourtellisASIT12}.

\citet{KourtellisASIT12} show that this algorithm can estimate, with probability
at least $1-1/n^2$, all the $k$-path betweenness values to within an additive
error $n^{-1/2+\alpha}/(n-1)$ using $2n^{1-2\alpha}k^2\ln n$ samples, for
$\alpha\in[-1/2,1/2]$. We can obtain the same guarantees with a \emph{much
lower} number $r$ of samples, precisely
\[
r=2n^{1-2\alpha}\left(\ln n+\frac{\lfloor\log_2(k-1)\rfloor+1}{2}\right)\enspace.
\]
In conclusion, we presented a tighter analysis of the algorithm
by~\citet{KourtellisASIT12}.

\subsection{Edge betweenness}
Until now we focused on computing the betweenness centrality of vertices. It is
also possible to define a betweenness centrality index for \emph{edges} of a
graph $G=(V,E)$~\citep{Anthonisse71,Brandes08}. Edge betweenness is useful, for
example, to develop heuristics for community detection~\citep{NewmanG04}. Given
$e\in E$, the edge betweenness $\mathsf{eb}(e)$ of $e$ is defined as the
fraction of shortest paths that \emph{contain} $e$, meaning that if $e=(u,v)$,
then a path $p$ contains $e$ if $u$ and $v$ appear consecutively in $p$.
Formally,
\[
\mathsf{eb}(e)=\frac{1}{n(n-1)}\sum_{p_{uv}\in\mathbb{S}_G}\frac{\mathds{1}_{p_{uv}}(e)}{\sigma_{uv}}, \forall e\in E\enspace.
\]

We can then define a range space $(\mathbb{S}_G,\mathcal{ER}_G)$ that contains
$|E|$ ranges $R_e$, one for each $e\in E$, where $R_e$ is the set of shortest
paths containing the edge $e$. By following the same reasoning as
Lemma~\ref{lem:vcdimuppbound} we have that
$\VC\left((\mathbb{S}_G,\mathcal{ER}_G)\right)\le\lfloor(\log_2(\VD(G) -
1)\rfloor + 1$. Using this result we can adapt the sample sizes for
Alg.~\ref{alg:algorithm} and~\ref{alg:topk} to compute good approximations of
betweenness for the (top-$k$) edges.

\fi

